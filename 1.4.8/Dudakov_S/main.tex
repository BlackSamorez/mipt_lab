\documentclass[a4, 12pt]{article}
\usepackage[a4paper,top=1.3cm,bottom=2cm,left=1.5cm,right=1.5cm,marginparwidth=0.75cm]{geometry}
\usepackage{setspace}
\usepackage{cmap}
\usepackage{mathtext}
\usepackage[utf8]{inputenc}
\usepackage[english,russian]{babel}
\usepackage[T2A]{fontenc}
\usepackage{multirow}
\usepackage{graphicx}
\usepackage{wrapfig}
\usepackage{tabularx}
\usepackage{float}
\usepackage{longtable}
\usepackage{hyperref}
\hypersetup{colorlinks=true,urlcolor=blue}
\usepackage[rgb]{xcolor}
\usepackage{amsmath,amsfonts,amssymb,amsthm,mathtools}
\usepackage{icomma}
\mathtoolsset{showonlyrefs=true}
\usepackage{euscript}
\usepackage{mathrsfs}

\DeclareMathOperator{\sgn}{\mathop{sgn}}
\newcommand*{\hm}[1]{#1\nobreak\discretionary{}
	{\hbox{$\mathsurround=0pt #1$}}{}}


\title{\textbf{Измерение модуля Юнга стержней методом акустического резонанса. (1.4.8)}}
\author{Дудаков Семён Б01-303}
\date{20 ноября 2023}


\begin{document}

	\maketitle

	\section{Введение}
    \textbf{Цель работы:} исследовать явление акустического резонанса в тонком стержне; из-
мерить скорость распространения продольных звуковых колебаний в тонких стержнях из различных материалов и различных размеров; измерить модули Юнга раз-
личных материалов.\\
\textbf{В работе используются:} генератор звуковых частот, частотомер, осциллограф,
электромагнитные излучатель и приёмник колебаний, набор стержней из различных материалов.

    \section{Теоретическая часть}
    Основной характеристикой упругих свойств твёрдого тела является его
модуль Юнга $E$. Согласно закону Гука, если к элементу среды приложено
некоторое механическое напряжение $\sigma$, действующее вдоль некоторой
оси $x$ (напряжения по другим осям при этом отсутствуют), то в этом эле-
менте возникнет относительная деформацию вдоль этой же оси
$\varepsilon = \Delta x/x_0$ , определяемая соотношением
\begin{equation}
    \sigma = \varepsilon E
\end{equation}
Если с помощью кратковременного воздействия в некотором элементе
твёрдого тела создать малую деформацию, она будет далее распростра-
няться в среде в форме волны, которую называют акустической или звуко-
вой. Распространение акустических волн обеспечивается за счёт упругости
и инерции среды. Волны сжатия/растяжения, распространяющиеся вдоль
оси, по которой происходит деформация, называются продольными. Как
будет строго показано далее, скорость $u$ распространения продольной аку-
стической волны в простейшем случае длинного тонкого стержня опреде-
ляется соотношением
\begin{equation}
    u = \sqrt{\frac{E}{\rho}}
\end{equation}
где $\rho$ — плотность среды.
Заметим, что размерность модуля Юнга $E$ равна [Н/м$^2$] и совпадает с
размерностью механического напряжения (или давления). Характерные
значения модуля Юнга металлов лежат в диапазоне $E\sim$ 1010 ÷ 1012 Па, так
что при плотности $\rho\sim$ 104 кг/м3 характерные значения скорости звука в
твёрдых телах составляют $u\sim$ 103 - 104 м/с.
В общем случае звуковые волны в твёрдых телах могут быть не только
продольными, но и поперечными — при этом возникает деформация сдвига
перпендикулярно распространению волны. Кроме того, описание распространения волн в неограниченных средах осложняется тем
обстоятельством, что при отличном от нуля коэффициенте Пуассона 1
напряжение вдоль одной из осей вызывает деформацию не только в про-
дольном, но и в поперечном направлении к этой оси. Таким образом, общее
описание звуковых волн в твёрдых телах — относительно непростая задача.
В данной работе мы ограничимся исследованием наиболее простого случая
упругих волн, распространяющихся в длинных тонких стержнях.
Рассмотрим стержень постоянного круглого сечения, радиус $R$ которого
много меньше его длины $L$. С точки зрения распространения волн стержень
можно считать тонким, если длина $\lambda$ звуковых волн в нём велика по срав-
нению с его радиусом: $\lambda R$. Такая волна может свободно распростра-
няться только вдоль стержня, поэтому можно считать, что стержень испы-
тывает деформации растяжения и сжатия только вдоль своей оси (заметим,
что в обратном пределе коротких волн $\lambda R$ стержень следует рассматри-
вать как безграничную сплошную среду). Если боковые стенки тонкого
стержня свободны (т.е. стержень не сжат с боков), то его деформации опи-
сывается законом Гука в форме (1), и, следовательно, его упругие свойства
определяются исключительно модулем Юнга среды.
Акустическая волна, распространяющаяся в стержне конечной длины $L$,
испытает отражение от торцов стержня. Если при этом на длине стержня
укладывается целое число полуволн, то отражённые волны будут склады-
ваться в фазе с падающими, что приведёт к резкому усилению амплитуды
их колебаний и возникновению акустического резонанса в стержне. Изме-
ряя соответствующие резонансные частоты, можно определить скорость
звуковой волны в стержне и, таким образом, измерить модуль Юнга мате-
риала стержня. Акустический метод является одним из наиболее точных
методов определения упругих характеристик твёрдых тел.
Получим дифференциальное уравнение, описывающее распростране-
ние упругих волн в тонком стержне.
Направим ось $x$ вдоль геометрической оси стержня (рис. 1). Разобьём
исходно недеформированный стержень на тонкие слои толщиной $\Delta x$. При
продольной деформации среды границы слоёв сместятся в некоторые но-
вые положения. Пусть плоскость среды, находящаяся исходно в точке $x$
\begin{figure}[H]
    \centering
    \includegraphics[scale = 1.5]{scheme.png}
    \caption{}
\end{figure}
сместилась к моменту $t$ на расстояние $\xi(x, t)$. Тогда слой, занимавший исходно
отрезок $[x, x+\Delta x]$, изменил свой продольный размер на величину $$\Delta \xi =
\frac{\partial \xi}{\partial x}\Delta x$$

\section{Методика измерений}
\begin{figure}[H]
    \centering
    \includegraphics[scale = 1.5]{stand.png}
\end{figure}
Схема экспериментальной установки приведена на рис. 3. Исследуемый
стержень 5 размещается на стойке 10. Возбуждение и приём колебаний в
стержне осуществляются электромагнитными преобразователями 4 и 6,
расположенными рядом с торцами стержня. Крепления 9, 11 электро-
магнитов дают возможность регулировать их расположение по высоте, а
также перемещать вправо-влево по столу 12.
\par Электромагнит 4 служит для возбуждения упругих механических про-
дольных колебаний в стержне. На него с генератора звуковой частоты 1 по-
даётся сигнал синусоидальной формы: протекающий в катушке электро-
магнита ток создаёт пропорциональное ему магнитное поле, вызывающее
периодическое воздействие заданной частоты на торец стержня (к торцам
стержней из немагнитных материалов прикреплены тонкие стальные
шайбы). Рядом с другим торцом стержня находится аналогичный электро-
магнитный датчик 6, который служит для преобразования механических
колебаний в электрические. Принцип работы электромагнитных датчиков
описан подробнее ниже.
Сигнал с выхода генератора поступает на частотомер 2 и на вход
канала X осциллографа 3. ЭДС, возбуждаемая в регистрирующем электро-
магните 6, пропорциональная амплитуде колебаний торца стержня, усили-
вается усилителем 7 и подаётся на вход канала Y осциллографа.
Изменяя частоту генератора и наблюдая за амплитудой сигнала с реги-
стрирующего датчика, можно определить частоту акустического резонанса
в стержне. Наблюдения в режиме X–Y позволяют сравнить сигналы гене-
ратора и датчика, а также облегчает поиск резонанса при слабом сигнале.
\par Как следует из формулы (2), модуль Юнга материала $E$ может быть
найден по скорости распространения акустических волн в стержне $u$ и его
плотности $\rho$. Для определения скорости $u$ в данной работе используется
метод акустического резонанса. Это явление состоит в том, что при часто-
тах гармонического возбуждения, совпадающих с собственными частотами
колебаний стержня $f \approx f_\text{рез}/Q$ , резко увеличивается амплитуда колебаний, при
этом в стержне образуется стоячая волна.
Возбуждение продольных колебаний в стержне происходит посред-
ством воздействия на торец стержня периодической силой, направленной
вдоль его оси. Зная номер гармоники $n$ и соответствующую резонансную
частоту $f_n$ , на которой наблюдается усиление амплитуды колебаний,
можно вычислить скорость распространения продольных волн в стержне:
\begin{equation}
    u = 2L\frac{f_n}{n}
\end{equation}

Таким образом, для измерения скорости $u$ необходимо измерить длину
стержня $L$ и получить зависимость резонансной частоты от номера резо-
нанса $n$. Если все теоретические предположения справедливы, эта за-
висимость будет прямой пропорциональностью.
Следует отметить, что в реальном металлическом стержне могут воз-
буждаться не только продольные, но и поперечные (в частности, изгибные)
колебания стержня. При этом каждому типу колебаний соответствует не
одна, а целый спектр частот. Таким образом, стержень «резонирует» не
только на частотах, определяемых формулой (15), но и на множестве дру-
гих частот. Для того чтобы отличить нужные нам резонансные частоты от
«паразитных», следует провести предварительные расчёты и не принимать
во внимание резонансы, не описываемые зависимостью (15).
Скажем также несколько слов о точности измерения резонансной ча-
стоты. В первую очередь отметим, что в идеальном случае резонанс дости-
гался бы при строгом совпадении частот $f = f_n$ (а амплитуда в резонансе
стремилась бы к бесконечности). Однако в реальности возбуждение стоя-
чей волны возможно при относительно малом отклонении частоты от резо-
нансной — амплитуда колебаний как функция частоты $A(f)$ имеет резкий
максимум при $f = f_n$.
\par Именно конечная ширина резонанса $\Delta f$ определяет в основном погреш-
ность измерения частоты в нашем опыте.
Используемые в работе металлические стержни являются весьма высо-
кодобротными системами: их добротность оказывается порядка $Q\sim$ 102 ÷
÷ 103 . Поэтому ширина резонанса оказывается довольно малой, что приво-
дит к необходимости тонкой настройки частоты генератора (при $f\sim$
 5 кГц ширина резонанса $\Delta f$ оказывается порядка нескольких герц)
Кроме того, время установления резонансных колебаний, которое можно
оценить как
\begin{equation}
    \tau_\text{уст} \sim \frac{1}{\Delta f} \sim \frac{Q}{f},
\end{equation}
оказывается весьма велико, из-за чего поиск резонанса нужно проводить, меняя частоту
генератора очень медленно.
\section{Оборудование}
Генератор звуковых частот, частотомер, осциллограф,
электромагнитные излучатель и приёмник колебаний, набор стержней из различных материалов.
\pagebreak
\section{Результаты измерений и обработка данных}

\subsection{Медь}
	Теоретическое значение первого резонанса равно
	\[f_1 = 3083.33 \text{ Гц}\] 
	Путем перестройки звукогого гененратора был найден первый резонанс $f_1 = 3158.35$ Гц. На экране наблюдается эллипс.
	Найдем частоты, на кратных гармониках: 
	\begin{center}
		\begin{tabular}{|c||c|c|c|c|c|}
			\hline
			Измерения (Гц) & 1 & 2 & 3 & 4 & 5\\
			\hline
			& 3158.35 & 6315.38 & 9478.32 & 12678.43 & 15821.73 \\
			\hline
		\end{tabular}
	\end{center}
	Найдем плотность медного стержня $d = (1.20 \pm 0.01) \text{ см}$, $l = (4.00 \pm 0.01) \text{ см}$, $m = (39.39 \pm 0.01) \text{ г}$.
	
	$\rho = (8.70 \pm 0.08)$ г/см$^3$.
 $\frac{d / 2}{l} = 9.9 * 10^{-3} << 1$ - стержень тонкий.
	
	Повторим опыты и для других стержней.
 \subsection{Дюраль}
	Частоты кратных гармоник: 
	\begin{center}
		\begin{tabular}{|c||c|c|c|c|c|c|c|c|}
			\hline
			Измерения (Гц) & 1 & 2 & 3 & 4 & 5 \\
			\hline
			& 4256.30 & 8568.28 & 12788.85 & 17085.65 & 21197.84 \\
			\hline
		\end{tabular}
	\end{center}
	
	Найдем плотность стержня из дюрали $d = (1.20 \pm 0.01) \text{ см}$, $l = (4.00 \pm 0.01) \text{ см}$, $m = (12.44 \pm 0.01) \text{ г}$.
	
	$\rho = (2.75\pm0.06)$ г/см$^3$.
 $\frac{d / 2}{l} = 9.6 * 10^{-3} << 1$ - стержень тонкий.
 
	
	\subsection{Сталь}
	Частоты кратных гармоник: 
	\begin{center}
		\begin{tabular}{|c||c|c|c|c|c|c|c|c|}
			\hline
			Измерения (Гц) & 1 & 2 & 3 & 4 & 5\\
			\hline
			& 4123.34 & 8250.22 & 12378.02 & 16505.05 & 20650.72 \\
			\hline
		\end{tabular}
	\end{center}
	
	Найдем плотность стержня из дюрали $d = (1.20 \pm 0.01) \text{ см}$, $l = (4.00 \pm 0.01) \text{ см}$, $m = (35.15 \pm 0.01) \text{ г}$.
	
	$\rho = (7.77 \pm 0.08)$ г/см$^3$.
 $\frac{d / 2}{l} = 10^{-4} << 1$ - стержень тонкий.
 \subsection{Добротность медного стержня}
	Напряжение при первом резонансе: $U = 13.5$ В.
	
	$\frac{U}{\sqrt{2}} = 9.5$ В.
	
	Частоты при таком напряжении: $\nu = 3163 \text{ Гц}$, $\nu = 3158 \text{ Гц}$
	
	$\Delta f = 10$ Гц.
	$Q = \frac{f}{\Delta f} = 315.8$
 \subsection{Скорость звука}
    Для вычисления скорости звука воспользуемся следующей формулой:
    \begin{equation}
		u = 2L\frac{f_n}{n}
		\label{sound spped}
	\end{equation}
	Для меди: $u = (3797.2 \pm 8.1)\text{ м/с}^2$ (0.21\%) \\
	Для дюрали: $u = (5087.4 \pm 6.1)\text{ м/с}^2$ (0.11\%) \\
	Для стали: $u = (4956.1 \pm 3.1)\text{ м/с}^2$ (0.06\%) \\
	(погрешность измерения длины стержня пренебрежительно мала, по сравнению с погрешностью $\frac{f_n}{n}$)
	
	\subsection{Определение модуля Юнга}
 Для вычисления модуля Юнга и его погрешности воспользуемся следующими формулами:
 \begin{equation}
		E = c_\text{ст}^2 \rho
		\label{Ung}
	\end{equation}
	\begin{equation}
		\Delta E = E \sqrt{4(\frac{\delta c_\text{ст}}{c_\text{ст}})^2 + (\frac{\delta\rho}{\rho})^2}
		\label{Ung acc}
	\end{equation}
	Для меди: $E = (125.4 \pm 7)\text{ ГПа}$ (5.5\%) \\
	Для дюрали: $E = (71.2 \pm 3)\text{ ГПа}$ (4.2\%) \\
	Для стали: $E = (190.8 \pm 7)\text{ ГПа}$ (4.8\%)
	\subsection{График частоты}
 \begin{center}
		\includegraphics[width=1\textwidth]{graphik.png}
	\end{center}
 Самая "крутая" прямая - дюраль\\
 Средняя - сталь\\
 Нижняя - медь\\
 \section{Обсуждение результатов}
    В ходе проделанной работы:
    \begin{enumerate}
        \item Была найдена добротность медного стержня при колебаниях
        \item Получили зависимость $f(n)$. Все полученные точки очень хорошо ложатся на прямые, что видно по представленному графику
        \item Была измерена скорость звука и найдены модули Юнга в предложенных для измерения материалах.
        \item В работе 1.3.1 мы измеряли модуль Юнга методом прогиба, и относительная погрешность получалась порядка 10\%. В данной же работе погрешность составляет 4-5\%, что в два раза точнее чем в 1.3.1, причем основной вклад в погрешность вносит именно измерение плотности.
    \end{enumerate}

    \section{Вывод}
    Подводя итоги, хотелось бы сказать, что с помощью метода акустического резонанса можно определить модуль Юнга в 2-2,5\% точнее(в данной работе). Погрешность получилась существенно меньше. Причем если очень аккуратно измерять плотность данных материалов ошибку можно уменьшить ещё в несколько раз.

\end{document}