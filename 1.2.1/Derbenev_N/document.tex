\documentclass[a4paper, 10pt]{article}%тип документа

%Русский язык
\usepackage[T2A]{fontenc} %кодировка
\usepackage[utf8]{inputenc} %кодировка исходного кода
\usepackage[english,russian]{babel} %локализация и переносы

\usepackage{multirow}

%Вставка картинок
\usepackage{graphicx}
\DeclareGraphicsExtensions{.pdf,.png,.jpg}

%Производные
\usepackage{physics}

%Математика
\usepackage{amsmath, amsfonts, amssymb, amsthm, mathtools}

\usepackage[left=10mm, top=20mm, right=18mm, bottom=15mm, footskip=10mm]{geometry}

%Заголовок
\author{Дербенев Никита Максимович}
\title{Лабораторная работа 1.2.1\\
	Определение скорости полета пули при помощи баллистического маятника}
\date{23 ноября 2023}
\begin{document}
	\maketitle
	\paragraph {Цель работы:}
		Определмить скорость полета пули, применяя законы сохранения и используя баллистические маятники
	\paragraph{В работе используются:}
	\begin{enumerate}
		\item Духовое ружье на штативе
		\item Осветитель
		\item Оптическая система для измерения отклонений маятника
		\item Измерительная линейка
		\item 10 пуль
		\item Весы
		\item Баллистические маятники
	\end{enumerate}
	\paragraph{Ход работы}
	\begin{enumerate}
		\item Измерим на весах массу каждой пули (табл. 1). Погрешность весов: $\sigma_m = 10$ мг
		\begin{table}[h]
			\centering
			\caption{Массы пуль}
			\begin{tabular}{|c|c|c|c|c|c|c|c|c|c|c|}
				\hline
				№ & 1 & 2 & 3 & 4 & 5 & 6 & 7 & 8 & 9 & 10 \\
				\hline
				$m_i$, мг & 504 & 518 & 504 & 498 & 500 & 510 & 508 & 508 & 510 & 500 \\
				\hline
			\end{tabular}
		\end{table}
		\item Измерим расстояние $L = (2197.5 \pm 1)$ мм.
		\item Соберем оптическую систему, включим и отстроим шкалу на ноль. Убедимся, что холостые выстрелы не влияют на маятник (движение не заметно глазом). Убедимся, что затухание колебаний незначительно (за 10 колебаний амплитуда уменьшается меньше, чем наполовину).
		\item Произведем 5 выстрелов пулями № 1-5, запишем амплитуду маятника в табл. 2:
		\begin{table}[h]
			\centering
			\caption{Результаты выстрелов в баллистический маятник №1}
			\begin{tabular}{|c|c|c|c|c|c|}
				\hline
				№ & 1 & 2 & 3 & 4 & 5 \\
				\hline
				$\Delta x$, мм & 9.5 & 8.75 & 9.0 & 9.75 & 9.5 \\
				\hline
				$L$, мм & \multicolumn{5}{|c|}{$2197.5 \pm 5$} \\
				\hline
				$m$, мг & 504 & 518 & 504 & 498 & 500 \\
				\hline
				$M$, г & \multicolumn{5}{|c|}{$2925 \pm 5$} \\
				\hline
				$u, \frac{\text{м}}{c}$ & 116.5 & 104.4 & 110.4 & 121.0 & 117.4 \\
				\hline
				$u_\text{ср}, \frac{\text{м}}{c}$ & \multicolumn{5}{|c|}{$114 \pm 11 (9.5\%)$} \\
				\hline
			\end{tabular}
		\end{table}
		\item Расчитаем начальную скорость пули по формуле:
		\[u=\dfrac{M}{m}\sqrt{\dfrac{g}{L}}\Delta x\]
		Расчитаем погрешность скорости:
		\[\varepsilon_{u_\text{сист}} = \varepsilon_M + \varepsilon_m + \tfrac{1}{2}\varepsilon_g + \tfrac{1}{2}\varepsilon_L + \varepsilon_{\Delta x} \approx 7.5\%\]
		\[\sigma_{u_\text{сист}} = u\varepsilon_{u_\text{сист}} \approx 8.5 \tfrac{\text{м}}{c}\]
		\[\sigma_{u_\text{случ}} \approx 6.6 \tfrac{\text{м}}{c}\]
		\[\sigma_u = \sqrt{\sigma_{u_\text{сист}}^2+\sigma_{u_\text{случ}}^2} \approx 10.8 \tfrac{\text{м}}{c}\]
		\[\varepsilon_u = \frac{\sigma_u}{u} \approx 9.5\%\]
		По итогу получаем:
		\[u = \left(114 \pm 11\right) \tfrac{\text{м}}{c} (9.5\%)\]
		
		\item Измерим для баллистического маятника №2 параметры $r$, $R$ и $d$ (табл. 3):
		\begin{table}[h]
			\centering
			\caption{Параметры баллистического маятника №2}
			\begin{tabular}{|c|c|c|c|}
				\hline
				Параметр & $r$ & $R$ & $d$ \\
				\hline
				Значение, мм & $213 \pm 2$ & $336.5 \pm 0.5$ & $1422 \pm 1$ \\
				\hline
			\end{tabular}
		\end{table}
		
		\item Измерим массы грузов: $M_1 = \left(725.5 \pm 0.1\right) \text{г}$, $M_2 = \left(738.7 \pm 0.1\right) \text{г}$
		\item Настроим осветительную систему. Включим осветитель и отстроим шкалу на ноль. Убедимся, что холостые выстрелы не влияют на маятник (движение не заметно глазом). Убедимся, что затухание колебаний незначительно (за 10 колебаний амплитуда уменьшается меньше, чем наполовину).
		
		\item Измерим период колебаний маятника $T1$ с грузом и $T2$ без (табл. 4):
		\begin{table}[h]
			\centering
			\caption{Период колебаний баллистического маятника №2}
			\begin{tabular}{|r||c|c|c|c|c||c|c|c|c|c|}
				\hline
				Опыт & \multicolumn{5}{|c||}{С грузами} & \multicolumn{5}{|c|}{Без грузов} \\
				\hline
				№ & 1 & 2 & 3 & 4 & 5 & 1 & 2 & 3 & 4 & 5 \\
				\hline
				$t$, c & 66.39 & 66.02 & 66.58 & 66.40 & 66.35 & 59.26 & 49.38 & 49.33 & 49.07 & 49.17 \\
				\hline
				$N$ & 10 & 10 & 10 & 10 & 10 & 12 & 10 & 10 & 10 & 10 \\
				\hline
				$T$, c & 6.64 & 6.60 & 6.66 & 6.64 & 6.64 & 4.94 & 4.94 & 4.93 & 4.91 & 4.92 \\
				\hline
				$T_\text{ср}$, c & \multicolumn{5}{|c||}{$6.64 \pm 0.02$} & \multicolumn{5}{|c|}{$4.93 \pm 0.01$} \\
				\hline
			\end{tabular}
		\end{table}
		\item Произведем 5 выстрелов пулями №6-10, запишем амплитуду маятника в табл. 5:
		\begin{table}[h]
			\centering
			\caption{Результаты выстрелов в баллистический маятник №2}
			\begin{tabular}{|c|c|c|c|c|c|}
				\hline
				№ & 1 & 2 & 3 & 4 & 5 \\
				\hline
				$\Delta x$, мм & 44 & 40 & 45 & 37.5 & 42.5 \\
				\hline
				$m$, мг & 510 & 508 & 508 & 510 & 500 \\
				\hline
				$u, \frac{\text{м}}{c}$ & 99.6 & 90.9 & 102.2 & 78.1 & 98.1 \\
				\hline
				$u_\text{ср}, \frac{\text{м}}{c}$ & \multicolumn{5}{|c|}{\bf{$94 \pm 11 \left(11.7\%\right)$}} \\
				\hline
			\end{tabular}
		\end{table}
		\item Расчитаем начальную скорость пули по формуле:
		\[u = \varphi\dfrac{\sqrt{kI}}{mr} \approx x\dfrac{\sqrt{kI}}{2dmr}\]
		\[\sqrt{kI} = \dfrac{4\pi MR^2T_1}{T_1^2-T_2^2} \approx (0.699 \pm 0.006) \tfrac{\text{кг}\cdot\text{м}^2}{c}\]
		\[\varepsilon_u = \varepsilon_{\sqrt{kI}} + \varepsilon_x + \varepsilon_m + \varepsilon_r \approx 5.8\%\]
		Расчитаем погрешность:
		\[\sigma_{u_\text{случ}} \approx 10 \frac{\text{м}}{c}\]
		\[\sigma_{u_\text{сист}} = u\varepsilon_u \approx 5 \frac{\text{м}}{c}\]
		\[\sigma_u = \sqrt{\sigma_{u_\text{сист}}^2+\sigma_{u_\text{случ}}^2} \approx 10.9 \tfrac{\text{м}}{c}\]
		По итогу:
		\[u = \left(94 \pm 11\right) \tfrac{\text{м}}{c} (11.7\%)\]
	\end{enumerate}
	\paragraph{Выводы:}
	С помощью баллистического маятника можно оценить начальную скорость пули из духового ружья. Этот метод при модификафии также подойдет и для огнестрельного оружия, если расположить балличтический маятник достаточно далеко от оружия, чтобы пороховые газы на него не влияли. Основной причиной погрешностей стало определение амплитуды маятника. Также разброс значений связан с тем, что начальная скорость пули при каждом выстреле действительно разная, для увеличения точности необходимо больше выстрелов.
\end{document}