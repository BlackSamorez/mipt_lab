\documentclass[a4paper, 12pt]{article}%тип документа


%отступы
\usepackage[left=1cm,right=1cm,top=1cm,bottom=2cm,bindingoffset=0cm]{geometry}

\usepackage{multirow}

%Русский язык
\usepackage[T2A]{fontenc} %кодировка
\usepackage[utf8]{inputenc} %кодировка исходного кода
\usepackage[english,russian]{babel} %локализация и переносы

%Вставка картинок
\usepackage{graphicx}
\graphicspath{{pictures/}}
\DeclareGraphicsExtensions{.pdf,.png,.jpg}

%Графики
\usepackage{pgfplots}
\pgfplotsset{compat=1.9}

%Математика
\usepackage{amsmath, amsfonts, amssymb, amsthm, mathtools}

%Заголовок
\author{Валеев Рауф Раушанович \\
группа 825}
\title{\textbf{Работа 1.4.5\\Изучение колебаний струны}}

\begin{document}
\maketitle
\newpage
\textbf{Схема установки}
Схема установки приведена на Рис. 3. Стальная гитарная струна 1 закрепляется в горизонтальном положении между двумя стойками с зажимами 2 и 3, расположенными на массивной станине 4. Один конец струны закреплен в зажиме 2 неподвижно. К противоположному концу струны, перекинутому че-рез блок, прикреплена платформа с грузами 5, создающими натяжение струны. Зажим 3 можно передвигать по станине, устанавливая требуемую длину струны. Возбуждение и регистрация колебаний струны осуществляются с помощью электромагнитных датчиков (вибраторов), расположенных на станине под струной. Электромагнитный датчик 6 подключен к звуковому генератору 7 и служит для возбуждения колебаний струны, частота которых измеряется с помощью частотомера 10 (в некоторых установках частотомер встроен в генератор). Колебания струны регистрируются с помощью электромагнитного датчика 8, сигнал с которого передается на вход осциллографа 9. Разъёмы, через которые датчики с помощью кабелей соединяются с генератором и осциллографом, расположены на корпусе станины.\\
\includegraphics[width=\textwidth]{145_2.png}
\textbf{Визуальное наблюдение стоячих волн}
\begin{enumerate}
\item Освобождаем зажим струны на стойке 3, установливаем длину струны $L = 50$ см. Натягиваем струну, поставив на платформу грузы ($F \approx $1 кг) (учитывая вес платформы и крепежа). Осторожно зажимаем струну в стойке, не деформируя струну. Возбуждающий датчик 6 должен располагаться рядом с неподвижной стойкой 2, т.е. вблизи узла стоячей волны.
\item Проводим предварительные расчёты. Оцениваем скорость распространения волн по формуле 
\[u = \sqrt{\dfrac{T}{\rho_1}}\] 
\begin{center}
\begin{tabular}{|c|c|c|c|}
\hline
$M_susp, g$ & $M_carg0, g$ & $\rho_1, g/m$ & $u, m/c$ \\ \hline
111,6       & 969,6        & 0,5684        & 137,9196 \\ \hline
\end{tabular}
\end{center}
\newpage
где  используя табличное значение плотности стали и приняв диаметр струны равным $d \approx 0,3$ мм. Для заданных значений длины струны и силы натяжения рассчитываем частоту основной гармоники $\nu_1$ согласно формуле
\[ \nu_n = \dfrac{u}{\lambda_n} = \dfrac{n}{2L} \sqrt{\dfrac{T}{\rho_1}}, n \in N\]
\item Включаем в сеть звуковой генератор и частотомер. Устанавливаем на генераторе тип сигнала — синусоидальный, частоту основной гармоники $\nu_1$ и максимальную амплитуду напряжения. При этом сигнал с выхода генератора должен быть подан на возбуждающий датчик 6 (проверяем правильность соединения проводов!)
\item Медленно меняя частоту звукового генератора в диапазоне $\nu = \nu_1 \pm 5$ Гц, добиваемся возбуждения стоячей волны на основной гармонике (одна пучность). Если при колебаниях струна касается регистрирующего датчика 8, осторожно сдвигаем датчик по скамье в сторону подвижного зажима струны 3. Определяем частоту первой гармоники по частотомеру.\\
\begin{center}
\includegraphics[width=0.6\textwidth]{145_1.png}
\end{center}
\item Увеличив частоту в 2 раза, получаем картину стоячих волн на второй гармонике, а затем и на более высоких гармониках. Обычно визуально удается наблюдать до 5-7 гармоник. Запишем значения частот $\nu_n$ стоячих волн, которые удастся пронаблюдать.
\begin{center}
\begin{tabular}{|c|c|c|}
\hline
\multicolumn{2}{|c|}{$f_{harmony},  Hz$} & \multirow{2}{*}{Номер} \\ \cline{1-2}
теория            & практика           &                        \\ \hline
137,92            & 137,8              & 1                      \\ \hline
275,84            & 278,6              & 2                      \\ \hline
413,76            & 418,9              & 3                      \\ \hline
551,68            & 560,2              & 4                      \\ \hline
689,6             & 690                & 5                      \\ \hline
827,52            & 830,8              & 6                      \\ \hline
965,44            & 974,5              & 7                      \\ \hline
\end{tabular}
\end{center}
\end{enumerate}
\textbf{Регистрация стоячих волн с помощью осцилогрофа}
\begin{enumerate}
\item Визуально настраиваем струну на основной гармонике, не меняя нагрузку струны и её длину. Устанавливаем регистрирующий датчик 8 в центре под струной (в пучности стоячей волны). Уменьшаем уровень выходного сигнала генератора так, чтобы при колебаниях струна не касалась датчика. Проверяем правильность соединения проводов. Сигнал колебаний струны с регистрирующего датчика 8 (основной сигнал) подается на вход канала CH2(Y) осциллографа. На вход канала CH1(X) подается опорный сигнал с генератора на частоте возбуждения струны.
\item Включите осциллограф в сеть. Для наблюдения колебаний струны в одноканальном режиме переключатель режима работы MODE блока вертикального отклонения должен стоять в положении CH2; тумблер режима работы канала Y — в положение AC; на блоке синхронизации устанавливаем SOURCE — CH2. Устанавливаем такие значения коэффициента усиления канала Y (VOLTS/DIV); постоянную времени развертки (TIME/DIV) и уровень синхронизации (LEVEL), чтобы на экране было удобно наблюдать форму сигнала.\\
Подстраиваем частоту $\nu$ генератора так, чтобы амплитуда сигнала была максимальна. Добиваемся отсутствия нелинейных искажений, уменьшая уровень возбуждения (амплитуду напряжения генератора) и подстраивая при этом частоту так, чтобы она соответствовала максимуму сигнала. Запишем окончательное значение частоты основной гармоники $\nu$.
\item Проводим измерение частот не менее 5 нечетных ($n = 1, 3, 5, 7, 9$) гармоник стоячих волн при длине струны 50 см и массе грузов $ \approx 1$ кг. Для наблюдения нечетных гармоник регистрирующий датчик следует размещать в центре под струной (как для основной гармоники).
\item Измеряем частоты четных ($n = 2, 4, ...$) гармоник. Для этого осторожно смещайте регистрирующий датчик 8 по станине в предварительно рассчитанные положения пучностей. Во избежание взаимного влияния («резонирования») датчиков регистрирующий датчик следует сдвигать в строну подвижного зажима струны 3.
\item Проведите опыты пп. 8 и 9 не менее, чем для пяти различных натяжений струны. При изменении нагрузки следует ослабить зажим струны в стойке 3, положить груз на чашку и вновь осторожно зажать струну. Максимальная нагрузка — не выше 3,5 кг!
\begin{center}
\begin{tabular}{|c|c|c|ccc}
\hline
\multicolumn{6}{|c|}{$M_susp = 111,6 g$, $\rho_1 = 0,5684 g/m$}                                                                           \\ \hline
теория          & практика       & Номер     & \multicolumn{1}{c|}{теория}   & \multicolumn{1}{c|}{практика} & \multicolumn{1}{c|}{Номер} \\ \hline
\multicolumn{3}{|c|}{$M_carg0 = 969,6 g$}    & \multicolumn{3}{c|}{$M_carg0 =  1944,2 g$}                                                 \\ \hline
137,9196        & 137,8          & 1         & \multicolumn{1}{c|}{190,1794} & \multicolumn{1}{c|}{188,5}    & \multicolumn{1}{c|}{1}     \\ \hline
275,8392        & 268            & 2         & \multicolumn{1}{c|}{380,3587} & \multicolumn{1}{c|}{384}      & \multicolumn{1}{c|}{2}     \\ \hline
413,7588        & 414,8          & 3         & \multicolumn{1}{c|}{570,5381} & \multicolumn{1}{c|}{567,6}    & \multicolumn{1}{c|}{3}     \\ \hline
551,6784        & 557,1          & 4         & \multicolumn{1}{c|}{760,7175} & \multicolumn{1}{c|}{776,5}    & \multicolumn{1}{c|}{4}     \\ \hline
689,598         & 693,7          & 5         & \multicolumn{1}{c|}{950,8968} & \multicolumn{1}{c|}{947,9}    & \multicolumn{1}{c|}{5}     \\ \hline
827,5176        & 835            & 6         & \multicolumn{1}{c|}{1141,076} & \multicolumn{1}{c|}{1165,3}   & \multicolumn{1}{c|}{6}     \\ \hline
965,4372        & 974,5          & 7         & \multicolumn{1}{c|}{1331,256} & \multicolumn{1}{c|}{1329}     & \multicolumn{1}{c|}{7}     \\ \hline
1103,357        & 1110,3         & 8         & \multicolumn{1}{c|}{1521,435} & \multicolumn{1}{c|}{1557,1}   & \multicolumn{1}{c|}{8}     \\ \hline
1241,276        & 1259           & 9         & \multicolumn{1}{c|}{1711,614} & \multicolumn{1}{c|}{1712,8}   & \multicolumn{1}{c|}{9}     \\ \hline
1379,196        & 1387,3         & 10        & \multicolumn{1}{c|}{1901,794} & \multicolumn{1}{c|}{1906}     & \multicolumn{1}{c|}{10}    \\ \hline
\multicolumn{3}{|c|}{$M_carg0 = 1460,8 g$}   & \multicolumn{3}{c|}{$M_carg0 =  2281,1 g$}                                                 \\ \hline
166,3238        & 164,1          & 1         & \multicolumn{1}{c|}{205,1715} & \multicolumn{1}{c|}{203,4}    & \multicolumn{1}{c|}{1}     \\ \hline
332,6477        & 335            & 2         & \multicolumn{1}{c|}{410,3431} & \multicolumn{1}{c|}{413,8}    & \multicolumn{1}{c|}{2}     \\ \hline
498,9715        & 495            & 3         & \multicolumn{1}{c|}{615,5146} & \multicolumn{1}{c|}{612,5}    & \multicolumn{1}{c|}{3}     \\ \hline
665,2953        & 670            & 4         & \multicolumn{1}{c|}{820,6861} & \multicolumn{1}{c|}{828,8}    & \multicolumn{1}{c|}{4}     \\ \hline
831,6192        & 826,4          & 5         & \multicolumn{1}{c|}{1025,858} & \multicolumn{1}{c|}{1022,2}   & \multicolumn{1}{c|}{5}     \\ \hline
997,943         & 1000           & 6         & \multicolumn{1}{c|}{1231,029} & \multicolumn{1}{c|}{1244,8}   & \multicolumn{1}{c|}{6}     \\ \hline
1164,267        & 1156,8         & 7         & \multicolumn{1}{c|}{1436,201} & \multicolumn{1}{c|}{1434,6}   & \multicolumn{1}{c|}{7}     \\ \hline
1330,591        & 1335           & 8         & \multicolumn{1}{c|}{1641,372} & \multicolumn{1}{c|}{1662,3}   & \multicolumn{1}{c|}{8}     \\ \hline
1496,914        & 1491,8         & 9         & \multicolumn{1}{c|}{1846,544} & \multicolumn{1}{c|}{1849,2}   & \multicolumn{1}{c|}{9}     \\ \hline
1663,238        & 1660           & 10        & \multicolumn{1}{c|}{2051,715} & \multicolumn{1}{c|}{2086}     & \multicolumn{1}{c|}{10}    \\ \hline
\end{tabular}
\begin{tabular}{|c|c|c|}
\hline
\multicolumn{3}{|c|}{$M_carg0  =  2773,7 g$} \\ \hline
225,3038         & 224           & 1         \\ \hline
450,6075         & 450,5         & 2         \\ \hline
675,9113         & 673           & 3         \\ \hline
901,2151         & 902,2         & 4         \\ \hline
1126,519         & 1122          & 5         \\ \hline
1351,823         & 1355,1        & 6         \\ \hline
1577,126         & 1573          & 7         \\ \hline
1802,43          & 1810,2        & 8         \\ \hline
2027,734         & 2026          & 9         \\ \hline
2253,038         & 2278          & 10        \\ \hline
\end{tabular}
\end{center}
\begin{tabular}{cccl}
\cline{1-3}
\multicolumn{1}{|c|}{$u, m/c$} & \multicolumn{1}{c|}{$T, H$} & \multicolumn{1}{c|}{$\sigma_u, m/c$} & \multirow{20}{*}{\includegraphics[width=0.7\textwidth]{145_6.jpg}} \\ \cline{1-3}
\multicolumn{1}{|c|}{139,7}    & \multicolumn{1}{c|}{10,6}   & \multicolumn{1}{c|}{0,9}             &                                                                                                         \\ \cline{1-3}
\multicolumn{1}{|c|}{166}      & \multicolumn{1}{c|}{15,4}   & \multicolumn{1}{c|}{1}               &                                                                                                         \\ \cline{1-3}
\multicolumn{1}{|c|}{191,4}    & \multicolumn{1}{c|}{20,1}   & \multicolumn{1}{c|}{3}               &                                                                                                         \\ \cline{1-3}
\multicolumn{1}{|c|}{207,8}    & \multicolumn{1}{c|}{23,4}   & \multicolumn{1}{c|}{2,2}             &                                                                                                         \\ \cline{1-3}
\multicolumn{1}{|c|}{226,9}    & \multicolumn{1}{c|}{28,3}   & \multicolumn{1}{c|}{1,7}             &                                                                                                         \\ \cline{1-3}
                               &                             &                                      &                                                                                                         \\
                               &                             &                                      &                                                                                                         \\
                               &                             &                                      &                                                                                                         \\
                               &                             &                                      &                                                                                                         \\
                               &                             &                                      &                                                                                                         \\
                               &                             &                                      &                                                                                                         \\
                               &                             &                                      &                                                                                                         \\
                               &                             &                                      &                                                                                                         \\
                               &                             &                                      &                                                                                                         \\
                               &                             &                                      &                                                                                                         \\
                               &                             &                                      &                                                                                                         \\
                               &                             &                                      &                                                                                                         \\
                               &                             &                                      &                                                                                                         \\
                               &                             &                                      &                                                                                                        
\end{tabular}

где $T1 = 10,6 H, T2 = 15,4 H, T3 = 20,1 H, T4 = 23,4 H, T5 = 28,3 H$ \\
\includegraphics[width=0.8\textwidth]{145_5.jpg}\\
Отсюда мы получаем, что $\rho_l = (5,5 \pm 0,1) \cdot 10^{-4} kg/m \approx 5,7 \cdot 10^{-4} kg/m$
\item Благодаря высокой добротности струны, возможно возбуждение её колебаний при кратных частотах генератора, меньших, чем $\nu_1$. Для наблюде-ния явления переключите осциллограф в режим (X-Y) и настройте установку на наблюдение основной гармоники. Затем уменьшите частоту возбуждения в два раза, установив на генераторе $\nu = 0.5 \nu_1$. На экране осциллографа должна наблюдаться фигура Лиссажу с одним самопересечением. \\
\includegraphics[width = 0.7\textwidth]{145_7.png}
\item Определите добротность $Q$ струны как колебательной системы, изме-рив её амплитудно-частотную характеристику (АЧХ) вблизи одной из резонансных частот (в качестве таковых ре комендуется выбрать $\nu_1$ или $\nu_3$) для не-скольких натяжений струны (по указа-нию преподавателя). Для расчётов воспользуйтесь извест-ным из теории колебаний результатом: добротность колебательной системы свя-зана с резонансной частотой $\nu_{res}$ и шириной резонансной кривой $\Delta \nu$ соотношением $Q = \dfrac{\nu_{res}}{\Delta \nu} \approx 918,7$ где ширина резонансной кривой $\Delta \nu$ из-меряется на уровне амплитуды, составляющей 0,7 от амплитуды в резонансе.\\
\includegraphics[width=0.3\textwidth]{145_3.png}\\
\begin{center}
\begin{tabular}{|c|c|c|c|c|cccccc}
\hline
        & 1      & 2      & 3      & 4      & \multicolumn{1}{c|}{5}      & \multicolumn{1}{c|}{6}      & \multicolumn{1}{c|}{7}      & \multicolumn{1}{c|}{8}      & \multicolumn{1}{c|}{9}      & \multicolumn{1}{c|}{10}     \\ \hline
$f, Hz$ & 137,8  & 137,78 & 137,77 & 137,75 & \multicolumn{1}{c|}{137,74} & \multicolumn{1}{c|}{137,72} & \multicolumn{1}{c|}{137,71} & \multicolumn{1}{c|}{137,7}  & \multicolumn{1}{c|}{137,68} & \multicolumn{1}{c|}{137,66} \\ \hline
$U, V$  & 0,036  & 0,034  & 0,032  & 0,03   & \multicolumn{1}{c|}{0,026}  & \multicolumn{1}{c|}{0,024}  & \multicolumn{1}{c|}{0,022}  & \multicolumn{1}{c|}{0,02}   & \multicolumn{1}{c|}{0,016}  & \multicolumn{1}{c|}{0,014}  \\ \hline
$U/U_0$ & 1      & 0,94   & 0,89   & 0,83   & \multicolumn{1}{c|}{0,72}   & \multicolumn{1}{c|}{0,67}   & \multicolumn{1}{c|}{0,61}   & \multicolumn{1}{c|}{0,56}   & \multicolumn{1}{c|}{0,44}   & \multicolumn{1}{c|}{0,39}   \\ \hline \hline
        & 11     & 12     & 13     & 14     & \multicolumn{1}{c|}{15}     & \multicolumn{1}{c|}{16}     & \multicolumn{1}{c|}{17}     & \multicolumn{1}{c|}{18}     & \multicolumn{1}{c|}{19}     & \multicolumn{1}{c|}{20}     \\ \hline
$f, Hz$ & 137,64 & 137,63 & 137,61 & 137,6  & \multicolumn{1}{c|}{137,58} & \multicolumn{1}{c|}{137,57} & \multicolumn{1}{c|}{137,55} & \multicolumn{1}{c|}{137,54} & \multicolumn{1}{c|}{137,52} & \multicolumn{1}{c|}{137,5}  \\ \hline
$U, V$  & 0,014  & 0,012  & 0,01   & 0,009  & \multicolumn{1}{c|}{0,009}  & \multicolumn{1}{c|}{0,008}  & \multicolumn{1}{c|}{0,008}  & \multicolumn{1}{c|}{0,007}  & \multicolumn{1}{c|}{0,006}  & \multicolumn{1}{c|}{0,006}  \\ \hline
$U/U_0$ & 0,39   & 0,33   & 0,28   & 0,25   & \multicolumn{1}{c|}{0,25}   & \multicolumn{1}{c|}{0,22}   & \multicolumn{1}{c|}{0,22}   & \multicolumn{1}{c|}{0,2}    & \multicolumn{1}{c|}{0,17}   & \multicolumn{1}{c|}{0,17}   \\ \hline \hline
        & 21     & 22     & 23     & 24     & \multicolumn{1}{c|}{25}     & \multicolumn{1}{c|}{26}     & \multicolumn{1}{c|}{27}     & \multicolumn{1}{c|}{28}     & \multicolumn{1}{c|}{29}     & \multicolumn{1}{c|}{30}     \\ \hline
$f, Hz$ & 137,49 & 137,47 & 137,46 & 137,82 & \multicolumn{1}{c|}{137,83} & \multicolumn{1}{c|}{137,85} & \multicolumn{1}{c|}{137,86} & \multicolumn{1}{c|}{137,88} & \multicolumn{1}{c|}{137,89} & \multicolumn{1}{c|}{137,91} \\ \hline
$U, V$  & 0,006  & 0,006  & 0,006  & 0,036  & \multicolumn{1}{c|}{0,034}  & \multicolumn{1}{c|}{0,03}   & \multicolumn{1}{c|}{0,026}  & \multicolumn{1}{c|}{0,024}  & \multicolumn{1}{c|}{0,022}  & \multicolumn{1}{c|}{0,018}  \\ \hline
$U/U_0$ & 0,17   & 0,17   & 0,17   & 1      & \multicolumn{1}{c|}{0,94}   & \multicolumn{1}{c|}{0,83}   & \multicolumn{1}{c|}{0,72}   & \multicolumn{1}{c|}{0,67}   & \multicolumn{1}{c|}{0,61}   & \multicolumn{1}{c|}{0,5}    \\ \hline \hline
        & 31     & 32     & 33     & 34     &                             &                             &                             &                             &                             &                             \\ \cline{1-5}
$f, Hz$ & 137,92 & 137,94 & 137,96 & 137,97 &                             &                             &                             &                             &                             &                             \\ \cline{1-5}
$U, V$  & 0,018  & 0,012  & 0,01   & 0,01   &                             &                             &                             &                             &                             &                             \\ \cline{1-5}
$U/U_0$ & 0,5    & 0,33   & 0,28   & 0,28   &                             &                             &                             &                             &                             &                             \\ \cline{1-5}
\end{tabular}
\end{center}

\includegraphics[width=\textwidth]{145_4.jpg}
\end{enumerate}
\end{document}