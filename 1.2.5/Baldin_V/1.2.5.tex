 \documentclass[a4paper,12pt]{article}
\usepackage[a4paper,top=1.3cm,bottom=2cm,left=1.5cm,right=1.5cm,marginparwidth=0.75cm]{geometry}
\usepackage{setspace}
\usepackage{cmap}
\usepackage{mathtext}
\usepackage[T2A]{fontenc}
\usepackage[utf8]{inputenc}
\usepackage[english,russian]{babel}
\usepackage{multirow}
\usepackage{graphicx}
\usepackage{wrapfig}
\usepackage{tabularx}
\usepackage{float}
\usepackage{longtable}
\usepackage{hyperref}
\hypersetup{colorlinks=true,urlcolor=blue}
\usepackage[rgb]{xcolor}
\usepackage{amsmath,amsfonts,amssymb,amsthm,mathtools}
\usepackage{icomma}
\mathtoolsset{showonlyrefs=true}
\usepackage{euscript}
\usepackage{mathrsfs}

\DeclareMathOperator{\sgn}{\mathop{sgn}}
\newcommand*{\hm}[1]{#1\nobreak\discretionary{}
	{\hbox{$\mathsurround=0pt #1$}}{}}

\title{\textbf{Исследование прецессии уравновешенного гироскопа (1.2.5)}}
\author{Балдин Виктор}
\date{6 ноября 2023}

\begin{document}

	\maketitle

	\section{Введение}

	\textbf{Цель работы:} исследовать вынужденную прецессию гироскопа, установить зависимость скорости вынужденной прецессии от величины момента сил, действующий на ось гироскопа и сравнить ее со скоростью, рассчитанной по скорости прецессии.\\
	\textbf{Оборудование:} гироскоп в кардановом подвесе, секундомер, набор грузов, отдельный ротор гироскопа, цилиндр известной массы, крутильный маятник, штангенсциркуль, линейка.

	\section{Теоретические сведения}

	В этой работе исследуется зависимость скорости прецессии гироскопа от момента силы, приложенной к его оси. Для этого к оси гироскопа подвешиваются грузы. Скорость прецессии определяется по числу оборотов рычага вокруг вертикальной оси и времни, которое на это ушло, определяемоу секундомером. В процессе измерений рычаг не только поворачивается в результате прецессии гироскопа, но и опускается. Поэтому его в начале опыта следует преподнять на 5-6 градусов.  Опять надо закончить, когда рычаг опустится на такой же угол.\\
	\begin{center}$
		\begin{array}{cc}
			\includegraphics[width=0.40\textwidth]{img1.png}&
			\includegraphics[width=0.40\textwidth]{img2.png}\\
		\end{array}$
	\end{center}

	Измерение скорости прецессии гироскопа позволяет вычислить угловую скорость вращения его ротора. Расчет производится по формуле:

	\begin{equation}
		\Omega = \frac{mgl}{I_z\omega_0},
	\end{equation}

	где $m$ -- масса груза, $l$ -- расстояние от центра карданова подвеса до точки крепления груза на оси гироскопа, $I_z$ -- момент инерции гироскопа по его главной оси вращения. $\omega_0$ -- частота его вращения относительно главной оси, $\Omega$ -- частота прецессии.\\
	Момент инерции ротора относительно оси симметрии $I_0$ измеряется по крутильным колебаниям точной копии ротора, подвешиваемой вдоль оси симметрии на десткой проволоке. Период крутильных колебаний $T_0$ зависит от момента инерции $I_0$ и модуля кручения проволоки $f$:

	\begin{equation}
		T_0 = 2\pi\sqrt{\frac{I_0}{f}}.
	\end{equation}

	Чтобы исключить модуль кручения проволоки, вместо ротора гироскопа к той же проволоке подвешивают цилиндр правильной формы с известными размерами и массой, для которого легко можно вычислить момент инерции $I_\text{ц}$. Для определения момента инерции ротора гироскопа имеем:

	\begin{equation}
		I_0 = I_\text{ц}\frac{T_0^2}{T_\text{ц}^2},
		\label{moment}
	\end{equation}
	Здесь $T_\text{ц}$ -- период крутильных колебаний цилиндра.\\
    \section{Методика измерений}
	\begin{center}
		\includegraphics[width=0.8\textwidth]{img3.png}
	\end{center}

	Скорость вращения ротора гироскопа можно определить и не прибегая к исследованию прецессии. У используемых в работе гироскопов статор имеет две обмотки, необходимые для быстрой раскрутки гироскопа. В данной работе одну обмотку искользубт для раскрутки гироскопа, а вторую -- для измерения числа оборотов ротора. Ротор электромотора всегда немного намагничен. Вращаясь, он наводит во второй обмотке переменную ЭДС индукции, частота которой равна частоте врещения ротора. Частоту этой ЭДС можно, в частности, измерить по фигурам Лиссажу, получаемым на экране осциллографа, если на один вход подать исследуемую ЭДС, а на другой -- переменное напряжение с хорошо прокалиброванного генератора. При совпадении частот на эеране получаем эллипс.
\end{document}
