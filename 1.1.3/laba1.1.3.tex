\documentclass{article}
\usepackage{amsmath}
\usepackage{mathtext}
\usepackage[T1,T2a]{fontenc}
\usepackage[utf8]{inputenc}
\usepackage[english, bulgarian, russian]{babel}
\usepackage{tikz}
\usepackage{pgfplots}

\title{Статистическая обработка результатов многократных измерений}
\date{2019-09-16}
\author{Панферов Андрей}

\begin{document}
\pagenumbering{gooble}
\maketitle
\newpage
\pagenumbering{arabic}
В работе используются: набор 240 сопротивленийб имеющих номинал 560 Ом, универсальный цифровой вольтметр B7-23, работающих в режиме "измерение сопротивлений постоянному току".

\begin{table}[h!]
\begin{center}
\caption{Результаты имерений сопротивлений 240 резисторов (в Омах):}
\begin{tabular}{||c|c|c|c|c|c|c|c|c|c|c|c||}
\hline
5026 & 5044 & 5079 & 4973 & 5052 & 4980 & 5162 & 5129 & 5118 & 5079 & 5103 & 5047 \\
\hline
5121 & 5079 & 5091 & 5109 & 5090 & 5044 & 5099 & 5168 & 4979 & 5149 & 5021 & 5115 \\
\hline
5135 & 5145 & 5091 & 5051 & 5134 & 5111 & 5072 & 5081 & 5109 & 5003 & 5086 & 5077 \\
\hline
5014 & 5085 & 5061 & 4930 & 5113 & 5076 & 5064 & 5072 & 5107 & 5061 & 5051 & 5091 \\
\hline
5066 & 5121 & 5031 & 5101 & 5084 & 5042 & 5053 & 5057 & 5010 & 5167 & 5047 & 5044 \\
\hline
5109 & 5059 & 5100 & 5097 & 5067 & 5063 & 5067 & 5144 & 5118 & 5119 & 5087 & 5140 \\
\hline
5059 & 5143 & 5079 & 5078 & 5206 & 5138 & 5093 & 5048 & 5147 & 5049 & 5053 & 5041 \\
\hline 
5137 & 5069 & 5006 & 5062 & 5067 & 5111 & 5118 & 5137 & 5068 & 5149 & 5023 & 5109 \\
\hline 
5144 & 5127 & 5102 & 5009 & 5121 & 5087 & 5139 & 5042 & 5163 & 5071 & 5159 & 5087 \\
\hline 
5052 & 5173 & 5049 & 5124 & 5074 & 5110 & 5179 & 5060 & 5065 & 5104 & 5023 & 5064 \\
\hline 
5097 & 5084 & 5075 & 5156 & 5139 & 5144 & 5079 & 5062 & 5108 & 4994 & 5067 & 5056 \\
\hline 
5139 & 4999 & 5079 & 5096 & 5109 & 5001 & 5061 & 5126 & 5104 & 5022 & 5076 & 5082 \\
\hline 
5119 & 5097 & 5093 & 5123 & 5067 & 5132 & 5071 & 5129 & 5207 & 5150 & 5134 & 5134 \\
\hline 
5071 & 5150 & 5159 & 5105 & 5048 & 5103 & 5140 & 5092 & 5073 & 5089 & 5145 & 5169 \\
\hline 
5127 & 5155 & 5070 & 5131 & 5099 & 5130 & 5103 & 5120 & 5102 & 5124 & 5154 & 5140 \\
\hline 
5073 & 5160 & 5008 & 5002 & 4996 & 5203 & 5086 & 5183 & 5140 & 5017 & 5095 & 5140 \\
\hline 
5141 & 5065 & 5149 & 5064 & 5036 & 5122 & 5142 & 5096 & 5095 & 4970 & 5123 & 5020 \\
\hline 
5067 & 5176 & 5051 & 5138 & 5011 & 5067 & 5143 & 4984 & 5006 & 5050 & 5153 & 5047 \\
\hline 
5081 & 5133 & 5005 & 5057 & 5105 & 5088 & 5139 & 5079 & 4986 & 5078 & 5055 & 5132 \\
\hline
5139 & 5076 & 5090 & 5098 & 5046 & 5129 & 5054 & 5029 & 5178 & 5140 & 5092 & 5154 \\
\hline
\end{tabular}
\end{center}
\end{table}





По этой таблице построимгистограммы для m=20 и m=30. Для удобства сравнения с нормальным распределением постороим его по формуле:
\begin{equation*}
y = \frac{N}{\sqrt{2\pi}\sigma}e^{\frac{(N-<N>)^2}{2\sigma^2}}
\end{equation*}

Рассчитаем плотность значений на участках длинны $\delta$R :

\begin{equation*}
\omega = \frac{\delta n}{\delta R} 
\end{equation*}


\begin{table}
\begin{right}
\caption{Серия 20 групп}
\end{right}
\begin{center}
\begin{tabular}{|c||c|c|c|c|c|c|c|c|c|c|}
\hline
k & 1 & 2 & 3 & 4 & 5 & 6 & 7 & 8 & 9 & 10 \\
\hline
$\delta$n & 22 & 28 & 19 & 24 & 24 & 10 & 5 & 0 & 3 & 0 \\
\hline
$\omega$*N & 1.47 & 1.87 & 1.27& 1.60 & 1.60 & 0.67 & 0.33 & 0.00 & 0.20 & 0.00 \\
\hline
\hline
k & -1&-2&-3&-4&-5&-6&-7&-8&-9&-10 \\
\hline
$\delta$n &33& 26& 16& 7& 9& 7& 4& 2& 0& 0\\
\hline
$\omega$*N & 2.20& 1.73& 1.07& 0.47& 0.60& 0.47& 0.27& 0.13& 0.00& 0.00 \\
\hline
\end{tabular}
\end{center}
\end{table}

Среднее значение R рассчитаем по формуле:

\begin{equation*}
<R> = \frac{1}{N}\sum^N_{i=1}R_i = 5090 Ом.
\end{equation*}

Среднеквадратичное отклонение находим по формуле:
\begin{equation*}
\sigma = \sqrt{\frac{1}{N}\sum^N_{i=1}(R_i-<R>)^2} \approx 49.3 Ом.
\end{equation*}


\begin{tikzpicture}[scale = 1.4]
\begin{axis}[
    axis lines = left,
    xlabel = {R, Ом},
    ylabel = {WN, 1/Ом},
    xmin=4930, xmax=5230,
    ymin=0, ymax=2.5
]
%Below the red parabola is defined
\addplot [
    domain=4940:5240, 
    samples=1000, 
    color=red,
]
{1.94217*exp(-(x-5090)*(x-5090)/4861)};
\addlegendentry{Нормальное  распределение}

\end{axis}

\begin{axis}[
	x tick label style={
		/pgf/number format/1000 sep=},
	enlargelimits=0,
	legend style={at={(0.1,-0.1)},
	anchor=north,legend columns=0},
	ybar interval=1,
	xmin=4930, xmax=5230,
    ymin=0, ymax=2.5,
    xtick={0},
    ytick={3}
]
\addplot 
	coordinates {(5080, 2.20)(5065, 1.73)(5050, 1.07)(5035, 0.47)(5020, 0.60)(5005, 0.47)(4990, 0.27)(4975, 0.13)(4960, 0.00)(4945, 0.00)(5095-15, 1.47)(5110-15, 1.87)(5125-15, 1.27)(5140-15, 1.60)(5155-15, 1.60)(5170-15, 0.67)(5185-15, 0.33)(5200-15, 0.00)(5215-15, 0.20)(5230-15, 0.00)};
\legend{Данные}
\end{axis}

\end{tikzpicture}



\newpage


\begin{table}
\begin{right}
\caption{Серия 10 групп}
\end{right}
\begin{middle}
\begin{tabular}{|c||c|c|c|c|c|c|c|c|c|c|}
\hline
k & 1 & 2 & 3 & 4 & 5 & -1 & -2 & -3 & -4 & -5 \\
\hline
$\delta$n & 50 & 43 & 34 & 5 & 3 & 59 & 23 & 16 & 6 & 0 \\
\hline
$\omega$*N & 1.67& 1.43& 1.13& 0.17& 0.10& 1.97& 0.77& 0.53& 0.20& 0.00 \\
\hline
\end{tabular}
\end{middle}
\end{table}


\begin{tikzpicture}[scale = 1.4]
\begin{axis}[
    axis lines = left,
    xlabel = {R, Ом},
    ylabel = {WN, 1/Ом},
    xmin=4930, xmax=5230,
    ymin=0, ymax=2.5
]
%Below the red parabola is defined
\addplot [
    domain=4940:5240, 
    samples=1000, 
    color=red,
]
{1.94217*exp(-(x-5090)*(x-5090)/4861)};
\addlegendentry{Нормальное  распределение}

\end{axis}

\begin{axis}[
	x tick label style={
		/pgf/number format/1000 sep=},
	enlargelimits=0,
	legend style={at={(0.1,-0.1)},
	anchor=north,legend columns=0},
	ybar interval=1,
	xmin=4930, xmax=5230,
    ymin=0, ymax=2.5,
    xtick={0},
    ytick={3}
]
\addplot 
	coordinates {(5065+15, 1.97) (5035+15, 0.77) (5005+15, 0.53) (4975+15, 0.20) (4945+15, 0.00) (5095-15, 1.67) (5125-15, 1.43) (5155-15, 1.13) (5185-15, 0.17) (5215-15, 0.10)};
\legend{Данные}
\end{axis}

\end{tikzpicture}


Видно, что гисторамма соответсвует теоретической зависимости. Теоретическая вероятность попадания измерений в интервал $<R>\pm\sigma$ равна 68$\%$, а в интервал $<R>\pm2\sigma$ равна 95$\%$.
Практически мы получаем, что велечина сопротивления резистора, наугад выбранного из данного набора, с вероятностью 70\% попадает в интервал $5080\pm50$ Ом, с вероятностью 95.8\% попадает в интервал $5080\pm100$ Ом, с вероятностью 99.6\% попадает в интервал $5080\pm150$ Ом.
Величины всех сопротивлений лежат в интервале $<R>\pm4\sigma$ 

\end{document}