\documentclass[a4paper, 12pt]{article}%тип документа

%отступы
\usepackage[left=2cm,right=2cm,top=2cm,bottom=3cm,bindingoffset=0cm]{geometry}

%Русский язык
\usepackage[T2A]{fontenc} %кодировка
\usepackage[utf8]{inputenc} %кодировка исходного кода
\usepackage[english,russian]{babel} %локализация и переносы

%Вставка картинок
\usepackage{wrapfig}
\usepackage{graphicx}
\graphicspath{{pictures/}}
\DeclareGraphicsExtensions{.pdf,.png,.jpg}

%оглавление
\usepackage{titlesec}
\titlespacing{\chapter}{0pt}{-30pt}{12pt}
\titlespacing{\section}{\parindent}{5mm}{5mm}
\titlespacing{\subsection}{\parindent}{5mm}{5mm}
\usepackage{setspace}

%Графики
\usepackage{multirow}
\usepackage{pgfplots}
\pgfplotsset{compat=1.9}

%Математика
\usepackage{amsmath, amsfonts, amssymb, amsthm, mathtools}

%Заголовок
\author{Валеев Рауф Раушанович \\
группа 825}
\title{\textbf{
}}
\newtheorem{task}{Задача}
\begin{document}
Тепловой поток в макроскопических системах традиционно описывается законом Фурье для диффузионного теплового потока. $ {} ^ {1} $ Развитие микро- и нанотехнологий привело в последние несколько лет к наблюдениям за недиффузионным тепловым потоком. Отклонения от диффузионного теплового потока могут возникать из-за волновой природы фононов, $ {} ^ {2} $ гидродинамических эффектов, $ {} ^ {3-10} $ и баллистического поведения. $ {} ^ {11-18} $ В нанопроволоках $ (\ mathrm {NWs}) $ рассеяние фононов на поверхности ННК снижает теплопроводность по сравнению с объемной. $ {} ^ {19} $ Однако, когда поверхность атомарно плоская, фононы могут зеркально отражаться от поверхности $ ^ {20,21} $ и может возникать баллистический тепловой поток в осевом направлении.
Реализация баллистического обтекания на большие расстояния при комнатной температуре требует большой длины свободного пробега фононов (MFP), что является сложной задачей из-за различных механизмов рассеяния. Лишь в нескольких исследованиях сообщалось о наблюдении баллистического транспорта при комнатной температуре. Сообщалось о баллистическом переносе при комнатной температуре на длину до $ 260 \ mathrm {nm} $ в коротких и широких графеновых лентах. $ {} ^ {12} $ Перенос тепла в короткие и узкие ленты диффузный из-за наличия краевого беспорядка. Баллистический тепловой поток при комнатной температуре на длину до $ 8.3 \ mu \ mathrm {m} $ был зарегистрирован для SiGe ННК с диаметрами в диапазоне $ 50-  200 \ mathrm {nm}. {} ^ {13} $ Его возникновение было объясняется отфильтровыванием высокочастотных коротких МФП-фононов рассеянием сплава. За пределами $ 8.3 \ mu \ mathrm {m} $ тепловой поток в этих $ \ mathrm {NW} $ оказался диффузным. Сообщалось также о переносе баллистических фононов при комнатной температуре в дырчатом кремнии длиной до $ 200 \ mathrm {нм}. {} ^ {15} $ В этом случае объяснение состояло в том, что поверхностный беспорядок отфильтровывает высокочастотные фононы, оставляя баллистический перенос низкочастотных фононов.
В этом письме мы демонстрируем осевой баллистический тепловой поток при комнатной температуре в ультратонких, диаметром $ 25 \ mathrm {nm} $, GaP ННК с длиной проволоки не менее $ 15 \ mu \ mathrm {m} $, а также резкий переход к диффузионному Тепловой поток при увеличении диаметра вдвое до $ 50 \ mathrm {nm} $ Баллистический перенос при комнатной температуре на такие большие расстояния и диффузионно-баллистический переход с уменьшающимся поперечным размером до сих пор не зарегистрированы. Температурный профиль вдоль $ \ mathrm {NWs} $ получен с помощью рамановской термометрии, подтверждающей наличие градиента температуры в диффузионном режиме потока и его отсутствие в баллистическом режиме. Мы выращиваем ННК GaP диаметром $ 25–140 $ нм методом газофазной эпитаксии из металлоорганических соединений, используя технику выращивания пар-жидкость-твердое тело (см. Дополнительный раздел SI1). ННК выращивают в виде массивов путем формирования рисунка подложка / катализатор. Внутри массива ННК имеют одинаковый диаметр по всей длине с разбросом диаметров $ <5 \ mathrm {nm} $. Мы анализируем структурные свойства ННК с помощью просвечивающей электронной микроскопии (ПЭМ). На ПЭМ-изображениях показаны ННК без дефектов упаковки, растущие вдоль направления \ langle 0001 \ rangle (см. Рисунок 1а). ННК имеют атомарно плоские боковые грани $ \ 11 \ overline {0} 0 \} $, покрытые слоем аморфного оксида примерно $ 2 \ mathrm {nm} $ (см. Рисунок $ 1 \ mathrm {~ b} $). Электронограммы, полученные в ПЭМ, также показывают, что все ННК с различными диаметрами, использованными в нашем исследовании, имеют кристаллическую структуру вюрцита (WZ) (см. Рисунок S1). Атомно-зондовая томография (АПТ) демонстрирует химическую чистоту ННК и отсутствие примесей тяжелых элементов (см. Рисунки $ 1 \ mathrm {c} $ и $ \ mathrm {S} 2 $). Следовательно, $ \ mathrm {NWs} $ атомарно чистые и бездефектные. Таким образом, наши GaP NW являются идеальными системами для баллистического переноса тепла. Ограничения метода APT не позволяют обнаружить оксидный слой. $ {} ^ {22} $ Мы производим микроприборы $ {} ^ {23,24} $ для изучения теплопроводности отдельных ННК (см. Раздел SI3). Устройства состоят из двух подвешенных мембран с платиновыми меандрами наверху, которые могут использоваться как нагреватель и термометр; см. изображение, полученное с помощью сканирующей электронной микроскопии (SEM) на рисунке 1d. Разное расстояние между мембранами позволяет измерять ННК разной длины. Используя микроманипулятор под оптическим микроскопом, мы помещаем выращенные одиночные ННК с выбранным диаметром между мембранами. Контакты $ \ mathrm {Pt} / \ mathrm {C} $ нанесены поверх ННК для улучшения теплового контакта (см. рисунок le). Мы определяем теплопроводность $ \ kappa $ из измеренной теплопроводности $ G $ и размеров ННК. Теплопроводность ННК, которая определяется как количество теплового потока, проходящего через материал при приложении разницы температур в один градус Кельвина, определяется путем подачи тока на одну из катушек и измерения температуры по обе стороны от ННК. Из этих измерений теплопроводность ННК извлекается путем расчета теплового баланса для чувствительной мембраны, где тепло, передаваемое через провод, приравнивается к теплу, которое покидает мембрану через $ \ mathrm {SiN} _ {x} $ балки, как это обычно делается при использовании такого рода устройств (см. раздел SI3 и ссылку 23.).

Извлеченный $ \ kappa $ при комнатной температуре $ T = 300 \ mathrm {~ K} $ показан на рисунке $ 2 \ mathrm {a} $ как функция диаметра $ d $ для $ \ mathrm {NWs} $ с приблизительно та же длина $ L = 6,7 \ pm 0,1 \ mu \ mathrm {m}, $ с $ d $ в диапазоне от 25 до $ 140 \ mathrm {nm}. $ Эти результаты были подогнаны, чтобы выявить влияние сопротивления теплового контакта, $ R_ {c} ^ {24} $ В СЗ с тепловым потоком, в котором преобладает диффузное рассеяние на поверхности СЗ, $ \ kappa $ увеличивается с $ d $ в соответствии с правилом Матиссена $ \ kappa \ propto $ $ \ left (1 + l_ { \ text {bulk}} / d \ right) ^ {- 1}, $ где $ l _ {\ text {bulk}} $ - это объемный $ \ mathrm {MFP} $, как показано пунктирной линией. Уменьшение $ \ kappa $ по отношению к этой линии, которое наблюдается для $ d \ geq 75 \ mathrm {nm}, $, указывает на то, что $ R_ {c} $ больше не является незначительным по сравнению с сопротивлением NW. $ {} ^ {24} $ Для тонких проволок $ (d $ $ <50 \ mathrm {nm}), $, напротив, наблюдается усиление $ \ kappa $, что указывает на то, что механизм недиффузионного переноса, в котором не доминирует контактное сопротивление становится все более важным с уменьшением диаметра. Чтобы исследовать механизм переноса в тонких проволоках, было построено тепловое сопротивление $ R $ $ = 1 / G $ как функция $ \ mathrm {NW} $ length $ L $ для диаметров $ d = 25 $ и $ 50 \ mathrm {nm} $ на рисунке $ 2 \ mathrm {~ b} $ (основная панель). Для $ 50 \ mathrm {nm} \ mathrm {NWs} R $ линейно зависит от $ L, $ в соответствии с диффузным тепловым потоком. Однако для $ 25 \ mathrm {nm} $ NW $ R $ не зависит от $ L $ для всех измеренных NW до $ L = 15 \ mu \ mathrm {m} $, что является доказательством баллистического переноса. Мы связываем это сопротивление с небольшим количеством баллистических режимов, способствующих транспортировке (см. Ниже). Отметим, что экстраполяция данных $ 50 \ mathrm {nm} $ проходит через начало координат, подтверждая пренебрежимо малое значение $ R_ {c}, $ в соответствии с моделью, в которой $ R_ {c} $ берется обратно пропорционально площади контакта. между контактом и NW (сплошная кривая на рисунке 2 а). Хотя мы не можем оценить $ R_ {c} $ для $ 25 \ mathrm {nm} $ ННК непосредственно из измерений, из этой модели следует пренебрежимо малое сопротивление контакта. Измеренная линейная зависимость проводимости устройства с количеством $ \ mathrm {NWs} $ для $ d = 25 \ mathrm {nm} $ и $ L = 6.7 \ pm 0.1 \ mu \ mathrm {m} $ (вставка на рис.
2б) исключает влияние возможной фоновой проводимости. Отметим, что эти данные были собраны с устройства с зазором между мембранами $ \ operatorname {SiN} _ {x} $ 4 $ \ mu \ mathrm {m}, $, где мы проставили $ \ mathrm {NWs } $ под углом для измерения $ \ mathrm {NWs} $ с $ L = 6.7 \ pm 0.1 \ mu \ mathrm {m}. $ Мы также охарактеризовали устройства с меньшим размером зазора $ 2 \ mu \ mathrm {m} $. Также для этих устройств можно исключить влияние возможной фоновой проводимости. Таким образом, данные однозначно указывают на переход от диффузного к независимому от длины баллистическому тепловому потоку в ННК диаметром 25 нм на беспрецедентных расстояниях. Чтобы дополнительно исследовать особенности перехода от диффузного к баллистическому, мы построили график на рисунке $ 2 \ mathrm {c} $ (основная панель) $ \ kappa $ против $ L $ для $ d = 25 $ и $ 50 \ mathrm {nm} \ mathrm {NWs} $ при $ 300 \ mathrm {~ K}, $ ясно демонстрирует тот замечательный факт, что для длинных проводов теплопроводность ультратонких 25 нм ННК превышает теплопроводность более толстых 50 нм ННК. На вставке показаны результаты для $ 40 \ mathrm {nm} \ mathrm {NWs} $ при 300 и $ 50 \ mathrm {~ K} $, показывая, что как при комнатной, так и при низкой температуре это переходный случай:
Тепловой поток является баллистическим до $ L \ приблизительно 11 \ mu \ mathrm {m} $, а затем становится диффузным. На рисунке $ 2 \ mathrm {\sim d} $ показана очень слабая зависимость теплопроводности $ G $ от $ T $ как для 25, так и для $ 50 \ mathrm {nm} $ ННК из $ 12.7 \ pm $ $ 0.2 \ mu \ mathrm {m} $. длиной примерно до $ 100 \ mathrm {\sim K} $ (ниже этой температуры G резко уменьшается из-за вымораживания фононов). Результаты для $ \ mathrm {NWs} $ 75 нм сравнимы с результатами для $ 50 \ mathrm {nm} \ mathrm {NWs} $, показывая также очень слабо зависящий от $ T $ диффузионный тепловой поток (см. Рисунок S6). Нечувствительность к T исключает объяснения диффузно-баллистического перехода на основе фонон-фононных взаимодействий, которые сильно зависят от T. В частности, рассеянию Umklapp, которое приводит к сильному уменьшению k в объемном GaP выше 50 K, 25, по-видимому, не играет существенной роли. Кроме того, участие фонон-поляритонов в качестве теплопроводных частиц маловероятно, поскольку они будут давать линейный по T вклад в G.26 Теплопроводность, показанная на рисунке 2, значительно меньше теплопроводности объемного GaP при комнатной температуре. . Для объемного GaP экспериментальные значения теплопроводности доступны только для равновесной структуры цинковой обманки со значением $ 75.2 при комнатной температуре \ mathrm {\sim W} / \ mathrm {m} \ cdot \ mathrm {K}. ^ {27} $ Расчетная объемная теплопроводность WZ GaP составляет $ 92 \ pm 5 $ $ \ mathrm {W} / \ mathrm {m} \ cdot \ mathrm {K} ^ {28} $.
Чтобы подтвердить баллистический перенос в тонком $ \ mathrm {NWs} $, мы исследуем тепловой профиль 25 и $ 50 \ mathrm {nm} \ mathrm {NWs} $ с помощью рамановской термометрии, которая является бесконтактным методом, на тех же устройствах. как используется для измерения теплопроводности (см. рисунок 3 а). Здесь локальная средняя температура решетки измеряется с помощью рамановского сдвига поперечной оптической (TO) моды $ {} ^ {29} $ (подробности эксперимента см. В разделе SI 6). Мы поддерживаем температуру системы на уровне $ T = 300 \ mathrm {~ K} $ и нагреваем горячий контакт на $ \ Delta T = 30 \ mathrm {~ K} $. Замена контактов дает те же результаты (см. Рис. $ \ Mathrm {S} 7 $). Сдвиг частоты $ \ delta \ omega $ ТО-моды (примерно $ 363 \ mathrm {~ cm} ^ {- 1} $ в GaP $) $ по отношению к случаю без применения разницы температур дает оценку повышение температуры локальной решетки, $ \ delta T, $, так как $ \ delta \ omega \ propto \ delta T $ (см. рис. $ 3 \ mathrm {~ b}, \ mathrm {~ d}) $. Это дифференциальное измерение нейтрализует изменения частоты режима ТО вдоль провода, вызванные другими эффектами, помимо разницы температур, такими как небольшие эффекты лазерного нагрева и локальные деформации. Здесь стоит подчеркнуть, что в случае стоксова рассеяния (как мы используем) измеренный спектр отражает среднюю температуру всех фононных мод, поскольку частотный сдвиг рамановского пика является результатом фононных эффектов более высокого порядка и, следовательно, включает все фононные моды. Другими словами, в наших рамановских экспериментах измеряется средняя температура решетки. Профиль температуры вдоль $ 50 \ mathrm {nm} \ mathrm {NW} $ (рис. $ 3 \ mathrm {c} $) является линейным, что соответствует диффузионному переносу. Важно отметить, что измерения рамановской термометрии на $ 50 \ mathrm {nm} $ NW дополнительно подтверждают отсутствие значительного контактного сопротивления, так как измеренный градиент температуры согласуется с приложенной разностью температур, которая была откалибрована независимым образом. Однако в $ 25 \ mathrm {nm} \ mathrm {NW} $ (Рисунок $ 3 \ mathrm {e}) $ мы наблюдаем симметричные скачки температуры на контактах и постоянную температуру по всему проводу с $ \ delta T $ close до $ 15 \ mathrm {~ K}, $ точно так, как предсказывает баллистическая теория переноса тепла. $ {} ^ {30} $ Отметим, что сама мода TO вносит незначительный вклад в тепловой поток из-за своего очень короткого MFP, но измеряет среднюю температуру теплоносителей фононов. В баллистическом тепловом потоке теплоносящие фононы, движущиеся в любом направлении через $ \ mathrm {NW} $, имеют температуру контакта, из которого они были инжектированы. Решетка термализуется одинаково с «горячими» и «холодными» фононами, и поэтому в режиме ТО проверяется среднее значение контактных температур, то есть $ \ delta T = 15 \ mathrm {~ K} $. Эти результаты демонстрируют аналогию с переносом электричества через баллистический квантовый провод, где электрический потенциал вдоль провода постоянен и равен среднему значению потенциалов, приложенных к контактам. $ {} ^ {31} $ Насколько нам известно, это первое измерение теоретически предсказанного профиля температуры для баллистического переноса тепла. Вместе с измерениями переноса тепла на рис. 2, это дает убедительное экспериментальное доказательство баллистического теплового потока. Интересно, что эти измерения намекают на существование остаточного рассеяния между теплоносителями и нетеплоносными фононами. 
Нам не известна существующая в литературе модель, которая могла бы зафиксировать диффузионно-баллистический переход для уменьшения размеров с масштабами длины, задействованными в этой работе. Чтобы дать предварительное объяснение, мы применим модель, основанную на формализме Ландауэра для переноса фононов. В нем не учитываются ангармонические эффекты, приводящие к фонон-фононному рассеянию, в соответствии со слабой зависимостью от $ T $ наших экспериментальных данных (см. Вставки к рисунку $ 2 \ mathrm {c}, \ mathrm {d} $). Применяемая нами модель была разработана Мерфи и Мур $ ^ {20} $ и Ченом и др. $ {} ^ {32} $ (см. Раздел $ \ mathrm {SI} 7 $) и успешно использовался для описания теплового потока в кремнии $ \ mathrm {NWs} ^ {32} $ и дырявом кремнии. $ {} ^ {15} $ Он различает два разных типа рассеяния фононов на поверхности ННК: поверхность кажется резкой (размытой) для фононов с обратной перпендикулярной компонентой волнового вектора, большей (меньшей), чем толщина поверхностного оксидного слоя, что приводит к к преимущественно зеркальному (диффузному) рассеянию с длинными (короткими) МФП. $ {} ^ {20,32} $ Это описание также согласуется с динамикой ходьбы Леви. $ {} ^ {18} $ Мы получаем количество мод $ N_ {s} $ и $ N _ {\ mathrm {d}} $ "зеркальных" и "диффузных" фононов на каждой частоте из расчетов ab initio фононная зонная структура WZ GaP и формирование фононных подзон в $ \ mathrm {NW} $ (см. раздел $ \ mathrm {SI} 8 $). На рисунке $ \ mathrm {S} 12 $ эти числа нанесены для $ d = 25 $ и $ 50 \ mathrm {nm} $ и толщины оксидного слоя $ h = 2.45 \ mathrm {nm} $ (подогнано для получения согласия с данные о теплопереносе $ d = 25 \ mathrm {nm} $, см. ниже. Это значение $ h $ близко к измеренной толщине (см. рисунок $ 1 \ mathrm {~ b} $). Отметим, что количество зеркальных фононные моды $ N _ {\ mathrm {s}} $ - это лишь малая часть $ (\ sim 1 \%) $ от числа диффузных мод $ N _ {\ mathrm {d}} $. Мы рассматриваем только фононы с частотой ниже дебаевской частоты.Высокочастотные фононы практически не возбуждаются при рассматриваемых температурах Нам не известна существующая в литературе модель, которая могла бы зафиксировать диффузионно-баллистический переход для уменьшения размеров с масштабами длины, задействованными в этой работе. Чтобы дать предварительное объяснение, мы применим модель, основанную на формализме Ландауэра для переноса фононов. В нем не учитываются ангармонические эффекты, приводящие к фонон-фононному рассеянию, в соответствии со слабой зависимостью от $ T $ наших экспериментальных данных (см. Вставки к рисунку $ 2 \ mathrm {c}, \ mathrm {d} $). Применяемая нами модель была разработана Мерфи и Мур $ ^ {20} $ и Ченом и др. $ {} ^ {32} $ (см. Раздел $ \ mathrm {SI} 7 $) и успешно использовался для описания теплового потока в кремнии $ \ mathrm {NWs} ^ {32} $ и дырявом кремнии. $ {} ^ {15} $ Он различает два разных типа рассеяния фононов на поверхности ННК: поверхность кажется резкой (размытой) для фононов с обратной перпендикулярной компонентой волнового вектора, большей (меньшей), чем толщина поверхностного оксидного слоя, что приводит к к преимущественно зеркальному (диффузному) рассеянию с длинными (короткими) МФП. $ {} ^ {20,32} $ Это описание также согласуется с динамикой ходьбы Леви. $ {} ^ {18} $ Мы получаем количество мод $ N_ {s} $ и $ N _ {\ mathrm {d}} $ "зеркальных" и "диффузных" фононов на каждой частоте из расчетов ab initio фононная зонная структура WZ GaP и формирование фононных подзон в $ \ mathrm {NW} $ (см. раздел $ \ mathrm {SI} 8 $). На рисунке $ \ mathrm {S} 12 $ эти числа нанесены для $ d = 25 $ и $ 50 \ mathrm {nm} $ и толщины оксидного слоя $ h = 2.45 \ mathrm {nm} $ (подогнано для получения согласия с данные о теплопереносе $ d = 25 \ mathrm {nm} $, см. ниже. Это значение $ h $ близко к измеренной толщине (см. рисунок $ 1 \ mathrm {~ b} $). Отметим, что количество зеркальных фононные моды $ N _ {\ mathrm {s}} $ - это лишь малая часть $ (\ sim 1 \%) $ от числа диффузных мод $ N _ {\ mathrm {d}} $. Мы рассматриваем только фононы с частотой ниже частоты Дебая .Высокочастотные фононы практически не возбуждаются при рассматриваемых температурах и имеют слишком короткие МФП, чтобы вносить значительный вклад в тепловой поток.

Модель, в которой вклады зеркальных фононов с сильно увеличенным MFP и диффузных фононов с $ \ mathrm {MFP} l = d $ рассматриваются одинаково, не согласуется с нашими данными (см. Рисунок $ \ mathrm {S} 13 $) . Вклад диффузных фононов в $ 25 \ mathrm {nm} \ mathrm {NWs} $ приведет к смещению теплопроводности $ \ kappa $, а вклад зеркальных фононов в $ 50 \ mathrm {nm} $ NWs будет приводят к линейному увеличению $ \ kappa $, ни чего не наблюдается. Подавление вклада в тепловой поток диффузных фононов в $ 25 \ mathrm {nm} \ mathrm {NWs} $, сопровождающееся усилением вклада зеркальных фононов, является в нашей модели необходимым ингредиентом для объяснения экспериментальных данных. Как показано сплошными линиями на рисунке $ 2 \ mathrm {~ b} - \ mathrm {d}, $ мы получаем согласие с экспериментом как для 25, так и для 50 $ \ mathrm {nm} \ mathrm {NWs} $, если мы:
(1) пренебречь для ННК $ 50 \ mathrm {nm} $ тепловым потоком, переносимым зеркальными фононами в выражении Ландауэра для $ G $ (eq $ \ mathrm {S} 4 $ в разделе $ \ mathrm {SI} 7) $ и (2) пренебрежение $ 25 \ mathrm {nm} \ mathrm {NWs} $ вкладом диффузных фононов с учетом только теплового потока, переносимого зеркальными фононами с сильно увеличенным MFP. Конечное, но сильно увеличенное MFP $ l $ до $ 100 \ mu \ mathrm {m} $ все равно будет соответствовать данным теплопереноса (пунктирные линии на рисунке $ 2 \ mathrm {~ b}, \ mathrm {c}) $ и будет приводят к небольшому температурному градиенту в $ 25 \ mathrm {nm} \ mathrm {NW} $, что находится в пределах экспериментальной неопределенности рамановских измерений (пунктирная линия на рисунке $ 3 \ mathrm {e}) $. Таким образом, фононы в $ 25 \ mathrm {nm} $ ННК перемещаются практически беспрепятственно и действительно могут быть названы баллистическими. Слабая зависимость теплопереноса от $ T $, показанная на рисунке $ 2 \ mathrm {~ d} $, объясняется тем, что основные особенности как в $ N _ {\ mathrm {s}} $, так и в $ N _ {\ mathrm {d }} $ появляются при энергиях фононов значительно ниже тепловой энергии (см. рисунок $ \ mathrm {S} 12 $). Хотя исходная модель Мерфи и Мура $ ^ {20} $ и Чен и др. $ {} ^ {32} $ делает различие между зеркальными и диффузными фононами, он не содержит механизма подавления теплового потока, переносимого диффузными фононами, и возникновения баллистического характера зеркальных фононов. В их модели зеркальные фононы все еще испытывают слабое рассеяние на диффузные фононные моды, вызванные неупорядоченной поверхностью ННК, что предотвращает баллистичность.
A possible cause for the suppression of the heat flow carried by diffusive phonons is Anderson localization ${ }^{33}$ of these phonons in the $25 \mathrm{nm}$ NWs. The phonon mean free paths of the diffusive phonons in the bulk WZ GaP crystal are in the range of hundreds of nanometers to several micrometers. The dominant scattering of these phonons is therefore by the amorphous surface of the NW. Recent works show that in NWs with rough or amorphous surfaces, Anderson localization could take place. ${ }^{34,35}$ The effect was shown to be increasingly important for thinner wires. ${ }^{20}$ Another cause could be the coupling of the localized modes in the amorphous oxide layer to propagating modes in the crystalline core. ${ }^{36,37}$ Such coupling could result in an effective layer near the boundary of the wire where the thermal conductivity is greatly reduced, an effect that can be much stronger for thinner wires. Finally, within the hydrodynamic framework, ${ }^{3-5,10}$ the so-called no-slip condition could also lead to a strong reduction of the thermal conductivity near the boundary of the $\mathrm{NW},$ quenching the contribution to the heat flow of diffusive phonons. A possible origin for the enhancement of the contribution of specular phonons and the emergence of ballisticity is a strongly reduced scattering of these phonons into diffusive modes caused by localization-induced decoupling of specular and diffusive phonons. This scenario, which would agree with the observed weak $T$ dependence of the transition, is explained in Section SI10.
These qualitative explanations of the reduction of the contribution of diffusive phonons to the heat flow and the enhancement of the contribution of specular phonons should be quantified, such that they can be experimentally verified. Ballistic phonon transport may find application in phonon transistors, ${ }^{38,39}$ phonon waveguides, ${ }^{40}$ and in novel cooling solutions for computer chips, which require rapid removal of heat from ever decreasing volumes. Ballistic heat flow could possibly be realized more easily in other compound semiconductors composed of lighter elements than Ga and $\mathrm{P}$, such as $\mathrm{Al}$ and $\mathrm{N}$, because of even longer phonon $\mathrm{MFPs}$. 

\end{document}
