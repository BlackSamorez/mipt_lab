\section{Приложение, экспериментальные данные}

В таблице \ref{app} указаны снятые показания для ртути (Hg) и неона (Ne), а также рассчитанные длины волн спектров водр=орода (H) и йода (I).

\begin{table}[!ht]\label{app}
    \centering
    \begin{tabular}{|l|l|l|l|}
    \hline
        Спектр & Номер линии & Угол барабана, deg & Длина волны, А \\ \hline
        Hg & 6 & 255 & 4047 \\ \hline
        ~ & 5 & 798 & 4358 \\ \hline
        ~ & 4 & 1457 & 4916 \\ \hline
        ~ & 3 & 1878 & 5461 \\ \hline
        ~ & 2 & 2057 & 5770 \\ \hline
        ~ & 1 & 2068 & 5791 \\ \hline
        Ne & 22 & 2098 & 5852 \\ \hline
        ~ & 21 & 2128 & 5882 \\ \hline
        ~ & 20 & 2146 & 5945 \\ \hline
        ~ & 19 & 2160 & 5976 \\ \hline
        ~ & 18 & 2184 & 6030 \\ \hline
        ~ & 17 & 2203 & 6074 \\ \hline
        ~ & 16 & 2212 & 6096 \\ \hline
        ~ & 15 & 2232 & 6143 \\ \hline
        ~ & 14 & 2242 & 6164 \\ \hline
        ~ & 13 & 2264 & 6217 \\ \hline
        ~ & 12 & 2282 & 6267 \\ \hline
        ~ & 11 & 2298 & 6305 \\ \hline
        ~ & 10 & 2310 & 6334 \\ \hline
        ~ & 9 & 2328 & 6383 \\ \hline
        ~ & 8 & 2337 & 6402 \\ \hline
        ~ & 7 & 2380 & 6507 \\ \hline
        ~ & 6 & 2413 & 6599 \\ \hline
        H & $\gamma$ & 768 & 4420 \\ \hline
        ~ & $\beta$ & 1405 & 5044 \\ \hline
        ~ & $\alpha$ & 2397 & 6546 \\ \hline
        I & 1 & 2081 & 5784 \\ \hline
        ~ & 1,5 & 1996 & 5622 \\ \hline
        ~ & гр & 1310 & 3972 \\ \hline
    \end{tabular}
    \caption{Экспериментальные данные}
\end{table}