\documentclass[12pt]{kiarticle} % You can learn about my document class "kiarticle" and install it to your device by following the link: https://github.com/Kiarendil/toolkitex
\graphicspath{{pictures/}}
\DeclareGraphicsExtensions{.pdf,.png,.jpg,.eps}
%%%
\pagestyle{fancy}
\fancyhf{}
%\renewcommand{\headrulewidth}{ 0.1mm }
\renewcommand{\footrulewidth}{ .0em }
\fancyfoot[C]{\texttt{\textemdash~\thepage~\textemdash}}
\fancyhead[L]{Лабораторная работа № 5.1 \hfil}
\fancyhead[R]{\hfil Иванов Кирилл, 625 группа }
\usepackage{multirow} % Слияние строк в таблице
\newcommand
{\un}[1]
{\ensuremath{\text{#1}}}
\newcommand{\eds}{\ensuremath{ \mathscr{E}}}
\newcommand{\ga}{\ensuremath{\gamma}}
\usepackage{tikz}
%%% Работа с таблицами
\usepackage{array,tabularx,tabulary,booktabs} % Дополнительная работа с таблицами
\usepackage{longtable}  % Длинные таблицы
\usepackage{multirow} % Слияние строк в таблице

\begin{document}
	
	\begin{titlepage}
		\begin{center}
			\large 	Московский физико-технический институт \\
			(государственный университет) \\
			Факультет общей и прикладной физики \\
			\vspace{0.2cm}
			
			\vspace{4.5cm}
			Лабораторная работа № 5.1 \\ \vspace{0.2cm}
			\large (Общая физика: квантовая физика) \\ \vspace{0.2cm}
			\LARGE \textbf{Измерение коэффициента ослабления потока $ \gamma $-лучей в веществе и определение их энергии}
		\end{center}
		\vspace{2.3cm} \large
		
		\begin{center}
			Работу выполнил: \\
			Иванов Кирилл,
			625 группа
			\vspace{10mm}		
			
		\end{center}
		
		\begin{center} \vspace{60mm}
			г. Долгопрудный \\
			2018 год
		\end{center}
	\end{titlepage}

	
	\paragraph*{Цель работы:} С помощью сцинтилляционного счетчика измерить линейные коэффициенты ослабления потока $ \gamma $-лучей в свинце, железе и алюминии; по их величине определить энергию $ \gamma $-квантов.
	
	\section{Теоретическое введение}
	
	Гамма-лучи возникают при переходе возбужденных ядер из одного энергетического состояния в другое, более низкое. Энергия $ \gamma $-квантов обычно заключена между несколькими десятками килоэлектронвольт и несколькими миллионами электрон-вольт. Гамма-кванты не несут электрического заряда, их масса равна нулю. Проходя через вещество, пучок $ \gamma $-квантов постепенно ослабляется. Ослабление происходит по экспоненциальному закону, который может быть записан в двух эквивалентных нормах:
	
	\begin{equation}\label{I(mu)}
	I = I_0 e^{-\mu l}, \quad I_o e^{-\mu 'm_1} 
	\end{equation}
	
	В этих формулах $ I, I_0 $ --- интенсивности прошедшего и падающего излучений, $ l $ --- длина пути, пройденного пучком \ga-лучей, $ m_1 $ ---
	масса пройденного вещества, приходящаяся на единицу площади, $ \mu $ и
	$ \mu' $ --- константы, величина которых зависит от вещества, сквозь кото-
	рое проходят \ga-лучи. Длину пути $ l $ обычно выражают в сантиметрах,
	поэтому $ \mu $ имеет размерность см$ ^{-1} $; величину $ m_1 $ измеряют в г/см$ ^2 $,
	так что размерность $ \mu' $ равна см$ ^2 $/г. Форма записи через массу является предпочтительной, потому что $ \mu' $, в отличие от $ \mu $, не зависит от плотности среды. 
	
	Ослабление потока \ga-лучей, происходящее при прохождении среды, связано с тремя эффектами: \textbf{фотоэлектрическим поглощением},
	\textbf{комптоновским рассеянием} и с \textbf{генерацией электрон-позитронных пар}. Рассмотрим эти эффекты.
	
	\subsection{Фотоэлектрическое поглощение.} При столкновении \ga-квантов с
	электронами внутренних атомных оболочек может происходить поглощение квантов. Энергия \ga-кванта передается соответствующему электрону, а импульс делится между этим электроном и оставшимся после
	его вылета ионом. Свободный электрон не может поглотить \ga-квант,
	так как при этом невозможно одновременно удовлетворить законам
	сохранения энергии и импульса. Наружные электроны не принимают участия в фотоэлектрическом поглощении, потому что они слабо
	связаны в атоме, так что их практически можно считать свободными.
	Вероятность $ dP_ф $ фотоэлектрического поглощения \ga-квантов пропорциональна длине пути $ dl $ и плотности электронов в среде (в расчет
	должны приниматься только электроны, принадлежащие внутренним
	оболочкам атомов):
	
	\begin{equation}\label{mu ph}
	dP_ф = \sigma_ф n_1 dl, \quad \mu_ф = \sigma_ф n_1
	\end{equation}
	
	Здесь $ n_1 $ --- плотность внутренних электронов, а $ \sigma_ф $ --- поперечное сечение фотоэлектрического поглощения. Поперечное сечение характеризует вероятность фотоэффекта, рассчитанную на один электрон. Связь между $ \mu_ф $ и $ \sigma_ф $ устанавливается из формулы \eqref{I(mu)} и в явном виде определяет зависимости $ \mu $ от плотности среды.
	
	Пусть в результате фотоэффекта энергия \ga-кванта передается
	электрону, находящемуся на $ i $-й оболочке атома. Обозначим через $ W_i $
	энергию связи этого электрона. После вылета из атома электрон приобретает кинетическую энергию $ T_i = \hbar \omega - W_i $.
	Освободившееся после вылета электрона место заполняется затем
	одним из электронов с вышележащих оболочек. При таких переходах
	возникает характеристическое рентгеновское излучение.
	
	\begin{wrapfigure}[13]{l}{0.4\linewidth}
		\includegraphics[width=\linewidth]{photo}
		\caption{Зависимость сечения фотоэффекта от энергии \ga-квантов}
		\label{ris photo}
	\end{wrapfigure}
	
	Вероятность фотоэффекта сложным образом зависит от энергии
	\ga-лучей и от заряда ядер. Для оценок можно пользоваться формулой
	
	\begin{equation}\label{sigma ph}
	\sigma_ф \propto \dfrac{Z^5}{(\hbar\omega)^{3,5}}
	\end{equation}
	
	Из формулы \eqref{sigma ph} видно, что вероятность фотоэффекта быстро возрастает при переходе от легких элементов к тяжелым резко падает с увеличением энергии \ga-квантов. На рис. \ref{ris photo} показана энергетическая зависимость сечения фотоэффекта. Из рисунка видно, что при энергиях \ga-квантов, лежащих в области атомных энергий связи, сечение претерпевает резкие изменения: при возрастании энергии это сечение скачкообразно возрастает, когда становится возможным выбивание электронов с очередной оболочки (на рис. \ref{ris photo} это скачки при энергиях $ W_M, W_L, W_K, $ соответствующих энергиям связи $ M, L $  и $ K $-электронов). В этой области сечение фотоэффекта очень велико по сравнению с сечениями других процессов. Поэтому фотоэффект является доминирующим механизмом поглощения \ga-квантов при не очень высоких энергиях.
	
	\subsection{Комптоновское рассеяние.} Комптоновским рассеянием (или комптоновским эффектом) называется упругое столкновение \ga-кванта с электроном. При таком столкновении \ga-квант передает электрону часть своей энергии, величина которой определяется углом рассеяния. В отличие от фотоэффекта, который может идти только на сильно связанных электронах, комптоновское рассеяние происходит на свободных или слабосвязанных электронах. Роль эффекта Комптона становится
	существенной только тогда, когда энергия квантов становится много
	больше энергии связи электронов в атоме (когда достаточно падает
	вероятность фотоэффекта). Атомные электроны в этом случае можно
	считать практически свободными, что обычно и делается при теоретическом анализе.
	
	Вероятность комптон-эффекта сложным образом зависит от энергии \ga-квантов. В том случае, когда энергия
	\ga-кванта много больше энергии покоя электрона, формула сильно
	упрощается, и выражение для сечения комптон-эффекта приобретает  вид:
	
	\begin{equation}\label{sigma k}
	\sigma_к = \pi r^2 \dfrac{mc^2}{\hbar\omega} \left( \ln{\dfrac{2\hbar\omega}{mc^2} + \dfrac{1}{2}} \right) 
	\end{equation}
	
	где $ r \simeq 2,8 \x 10^{13} $ --- классический радиус электрона,$ m $ --- его масса. Из формулы \eqref{sigma k} следует, что сечение комптон-эффекта с ростом энергии фотонов падает далеко не так резко, как сечение фотоэффекта.
	Сечение $ \sigma_к $ относится к одному свободному электрону, в то время как приведенное выше сечение фотоэффекта \eqref{sigma ph} рассчитано на атом.
	Комптоновское рассеяние, отнесенное к атому, оказывается, естественно, в $ Z $ раз больше. 
	
	Комптоновский коэффициент линейного ослабления $ \mu_к $ связан с
	сечением $ \sigma_к $ формулой, аналогичной \eqref{mu ph}. Под $ n $ следует в этом случае понимать плотность слабо связанных электронов, т. е. практически полную плотность электронов в веществе.
	Отметим в заключение, что, в отличие от фотоэффекта, эффект
	Комптона приводит не к поглощению \ga-квантов, а к их рассеянию и
	уменьшению их энергии.
	
	\subsection{Образование пар}
	
	 При энергиях \ga-лучей, превышающих $ 2mc^2 = 1,02  $МэВ, становится возможен процесс поглощения \ga-лучей, связанный с образованием электрон-позитронных пар. Рождение пар не
	может происходить в вакууме, оно возникает в электрическом поле
	ядер. Вероятность этого процесса приблизительно пропорциональна
	$ Z^2  $ и сложным образом зависит от энергии фотона. При энергиях больше $ 2mc^22  $ фотоэффект даже для самых тяжелых ядер уже не играет
	практически никакой роли. Вероятность образования пар должна поэтому сравниваться с вероятностью комптоновского рассеяния. При
	энергиях, с которыми приходится иметь дело при изучении ядер, рождение пар существенно только в самых тяжелых элементах. Так, даже
	для свинца вероятность рождения пар сравнивается с вероятностью
	комптоновского эффекта только при энергии около 4,7 МэВ.
	
	\subsection{Полный коэффициент ослабления \ga-лучей}
	
	Полный линейный коэффициент $ \mu $ ослабления пучка \ga-квантов при прохождении через вещество равен сумме коэффициентов для всех трех рассмотренных процессов. На рис. \ref{ris mu} изображены графики $ \mu $ для различных материалов.
	
	\begin{figure}[h!]
		\centering
		\includegraphics[width=0.6\linewidth]{mu}
		\caption{Полные коэффициенты ослабления потока \ga-лучей в алюминии, железе и свинце}
		\label{ris mu}
	\end{figure}
	
	
	Обратимся вновь к формуле \eqref{I(mu)}. Ее нетрудно получить из теоретических соображений. Рассмотрим опыты, поставленные в хорошей
	геометрии, т. е. в условиях, когда исследуется прохождение сквозь вещество узкого параллельного пучка \ga-лучей. В этом случае не только
	фотоэлектрическое поглощение и генерация пар, но и комптоновское
	рассеяние выводит \ga-кванты из пучка.
	Поэтому при прохождении через вещество меняется только количество, но не энергия \ga-квантов в пучке, так что коэффициент $ \mu $, характеризующий поглощение \ga-квантов в веществе, не зависит от длины
	пути. Обозначим через $ -dN $ число \ga-квантов, выбывших из пучка на
	пути $ dl $. Это число пропорционально имеющемуся их числу $ N $ и прой-
	денному пути $ dl $. Имеем, следовательно,
	
	\begin{equation}\label{N}
	-dN = \mu N dl \te N = N_0 e^{-\mu l}
	\end{equation}
	
	т.е то же самое, что и формула \eqref{I(mu)}. В плохой геометрии, когда рассеянные под небольшими углами
	\ga-кванты остаются в пучке, их спектр с прохождением вещества меняется, и формула \eqref{I(mu)}, вообще говоря, неприменима. Эта формула, однако, работает и в этом случае лучше, чем можно было бы ожидать. Причина хорошего согласия заключается в том, что \ga-кванты с энергией 1 -- 2 МэВ, потерявшие энергию из-за комптоновского рассеяния,
	быстро выбывают из пучка из-за резкого увеличения сечений $ \sigma_ф $ и $ \sigma_к $.
	
	В данной работе коэффициент ослабления $ \mu $ измеряется в хорошей
	геометрии. Из формулы \eqref{I(mu)} или \eqref{N} имеем
	
	\begin{equation}\label{mu}
	\mu = \dfrac{1}{l} \ln{\dfrac{N_0}{N}}
	\end{equation}
	
	Для определения коэффициента ослабления нужно, таким образом, измерить толщину образца $ l $, число падающих частиц $ N_0 $ и число
	частиц $ N $, прошедших через образец.
	
	\section{Экспериментальная установка}
	
		\begin{figure}[h!]
		\centering
		\includegraphics[width=0.7\linewidth]{lab}
		\caption{Блок-схема установки, используемой для измерения коэффициентов ослабления потока \ga-лучей: И --- источник \ga-лучей; $ Pb $ --- свинцовый контейнер с коллиматорным каналом; П --- набор поглотителей; С --- сцинтиллятор (кристалл
			NaI(Tl) ); Ф --- формирователь-выпрямитель}
		\label{ris lab}
	\end{figure}

Схема установки, используемой в работе, показана на рис. \ref{ris lab}. Свинцовый коллиматор выделяет узкий почти параллельный пучок \ga-квантов, проходящий через набор поглотителей П и регистрируемый сцинтилляционным счетчиком). Сигналы от счетчика усиливаются и регистрируются пересчетным прибором ПП. Высоковольтный выпрямитель ВВ обеспечивает питание сцинтилляционного счетчика.

При недостаточно хорошей геометрии в результаты опытов могут
вкрасться существенные погрешности. В реальных установках всегда имеется конечная вероятность того, что \ga-квант провзаимодействует в
поглотителе несколько раз до того, как попадет в детектор. Чтобы уменьшить число таких случаев, в данной работе сцинтилляционный счетчик расположен на большом расстоянии от источника \ga-квантов, а поглотители имеют небольшие
размеры. Их следует устанавливать за коллиматорной щелью на некотором расстоянии друг от друга, чтобы испытавшие комптоновское
рассеяние и выбывшие из прямого потока кванты с меньшей вероятностью могли в него вернуться.

\section{Выполнение работы}

 \textit{\textbf{Примечание}: в работе через $ N' $ обозначается число частиц за какое-то время $ t $ (безразмерная величина), $ N = \dfrac{N'}{t} $ --- число частиц за секунду [с$ ^{-1} $]. Погрешность этих величин считается как} 
 
 \begin{equation}\label{}
 \sigma_N = N \x \dfrac{\sqrt{N'}}{N'} = \dfrac{\sqrt{N'}}{t}
 \end{equation}

Включив установку, для начала перекроем коллиматорный канал толстой свинцовой пробкой и в течении 10 минут измерим фон.Получаем $ N'ф = 14939 $ частиц, т.е. $ N_ф = 24,9 \pm 0,2 \; с^{-1} $.

Теперь измерим число частиц без поглотителя: за 20 секунд счетчик показывает $ N'_0 = 244487 $ частиц, т.е. $ N_0 = 12199 \pm 25 \; с^{-1}$ (с учетом вычета фона). 

Теперь проведем 4 серии измерений, закрывая коллиматорный канал поглотителями из свинца, железа, алюминия и пробки (древесины) разной толщины. Результаты занесем в таблицы \ref{table_1}-\ref{table_4}. 

При измерении длины поглотителя мы измеряли каждый "<кусочек"> с помощью штангенциркуля и прибавляли длину "<кусочка"> к уже имеющейся длине поглотителя. Таким образом,  погрешность измерения длины $ l $ мы оцениваем как

\begin{equation}\label{}
\sigma_{l_i} = \sqrt{\sum\limits_{j=1}^i \sigma_{l_j}^2} = \sqrt{i}\sigma_ш
\end{equation}

Где $ \sigma_{l_j} = \sigma_ш = 0,2 \;мм $ --- погрешность измерения каждого "<кусочка">. 

Из формулы \eqref{mu} выведем удобную для фитирования полученных данных линейной функцией формулу 

\begin{equation}\label{yotx}
\ln N = \ln N_0 - \mu l
\end{equation}

Построим для каждого образца график вида $ y = b - ax $, результаты фита сведем в таблицу \ref{table_5}.

С помощью коэффициентов из таблицы \ref{table_5} и графика рис. \ref{ris mu} определим энергию \ga-квантов:

\begin{itemize}
	\item У свинца $ \mu \approx  0,969 \; см^{-1} \te E \simeq 0,78 $ МэВ. Табличное значение для энергии 0,8 МэВ --- 0,982 $ см^{-1}  $.
	\item У железа $ \mu \approx  0,528 \; см^{-1} \te E \simeq 0,81 $ МэВ. Табличное значение для энергии 0,8 МэВ --- 0,526 $ см^{-1}  $.
	\item У алюминия $ \mu \approx  0,205 \; см^{-1} \te E \simeq 0,66 $ МэВ. Табличное значение для энергии 0,8 МэВ --- 0,184 $ см^{-1}  $, для энергии 0,6 МэВ --- 0,210 $ см^{-1}  $.
\end{itemize} 

Таким образом, средняя энергия \ga - квантов равна 0,75 МэВ.


 \begin{table}[h!]
	\caption{Результаты фита $ y = b - ax $}
	\begin{center}
		\begin{tabular}{|c|c|c|c|}
			\hline
			Образец  &  $ b = \ln N_0 $  & $ a = \mu, \; см^{-1} $ & $ \chi^2_\nu $ \\ \hline
			Свинец  & $ 9,25	\pm	0,02 $ &   $ 0,969	\pm	0,014 $   &   0,7    \\
			Железо  & $ 9,29	\pm	0,02 $ &   $ 0,528	\pm	0,006 $   &    1,5        \\
			Алюминий & $ 9,38	\pm	0,04 $ &   $ 0,205	\pm	0,005 $   &   24         \\
			Пробка  & $ 9,39	\pm	0,03 $ &  $ 0,0095	\pm	0,0004 $  &    3,6        \\ \hline
		\end{tabular}
	\end{center}
	\label{table_5}
\end{table}

\section{Вывод}

В этой работе мы изучили ослабление потоков \ga-квантов в четырех различных веществах: свинце, железе, алюминии и пробке (древесине) и экспериментальным путем опередили коэффициенты ослабления (таблица \ref{table_5}). 

С помощью этих коэффициентов мы определили энергию \ga-квантов --- примерно 0,75 МэВ.

\subsection*{Радиационная безопасность}

В ходе выполнения данной работы нами также были проведены небольшие измерения по радиационной безопасности в лаборатории.

Проведём измерения уровня радиации в нескольких местах лаборатории вдали от радиоактивных источников; измеренный уровень радиации составляет 0,15 мкЗв/ч и 0,18 мкЗв/ч.

Измеренный уровень радиации на рабочем месте — 0,12 мкЗв/ч.

Измеренная интенсивность излучения вблизи свинцовой защиты радиоактивного источника — 0,22 мкЗв/ч. При измерении интенсивности излучения напротив коллиматора вблизи выходящего пучка дозиметр зашкаливает.

В работе использовался источник $\gamma$ - излучения, поэтому эквивалентная доза совпадает с поглощённой. 

Полученная за время работы в лаборатории доза облучения: 0,11 мкЗв/ч $\cdot$ 3 ч = 0,36 мкЗв. Предельно допустимая эффективная доза составляет 5 мЗв за год и 1 мЗв в год в среднем за любые последовательные 5 лет. Следовательно, полученная за время работы в лаборатории доза облучения значительно меньше предельно допустимой.

\section*{Графики и таблицы}

\begin{figure}[h!]
	\label{graf_pb}
	\includegraphics[scale=0.47]{pb.pdf}
	\caption{Ослабление \ga-лучей в \textbf{свинце}}
\end{figure}

\begin{figure}[h!]
	\label{graf_fe}
	\includegraphics[scale=0.47]{fe.pdf}
	\caption{Ослабление \ga-лучей в \textbf{железе}}
\end{figure}

\begin{figure}[h!]
	\label{graf_al}
	\includegraphics[scale=0.47]{al.pdf}
	\caption{Ослабление \ga-лучей в \textbf{алюминии}}
\end{figure}

	\begin{figure}[h!]
	\label{graf_d}
	\includegraphics[scale=0.47]{d.pdf}
	\caption{Ослабление \ga-лучей в \textbf{пробке (древесине)}}
\end{figure}


\begin{table}[h!]
	\caption{Ослабление \ga-лучей в \textbf{свинце}}
	\begin{center}
		\begin{tabular}{|c|c|c|c|c|c|c|c|c|}
			\hline
			№ & $ l $, мм & $ \sigma_l $, мм & $ N' $ & $ t $, с & $ N, \; с^{-1}$  & $ \sigma_N, \; с^{-1}$ & $ \ln N $ & $ \sigma_{\ln N} $ \\
			\hline
			1 & 4.55 & 0.2 & 137190 & 20 & 6835 & 19 & 8.83 & 0.003 \\
			2 & 8.95 & 0.28 & 175029 & 40 & 4351 & 10 & 8.378 & 0.002 \\
			3 & 13.25 & 0.35 & 119005 & 40 & 2950 & 9 & 7.99 & 0.003 \\
			4 & 18.15 & 0.4 & 105458 & 60 & 1733 & 5 & 7.457 & 0.003 \\
			5 & 23.15 & 0.45 & 101334 & 90 & 1101 & 4 & 7.004 & 0.003 \\
			6 & 28.05 & 0.49 & 109744 & 150 & 707 & 2 & 6.561 & 0.003 \\
			\hline
		\end{tabular}
	\end{center}
	\label{table_1}
\end{table}

\begin{table}[h!]
	\caption{Ослабление \ga-лучей в \textbf{железе}}
	\begin{center}
		\begin{tabular}{|c|c|c|c|c|c|c|c|c|}
			\hline
			№ & $ l $, мм & $ \sigma_l $, мм & $ N' $ & $ t $, с & $ N, \; с^{-1}$  & $ \sigma_N, \; с^{-1}$ & $ \ln N $ & $ \sigma_{\ln N} $ \\
			\hline
			1 & 10.4 & 0.2 & 190381 & 30 & 6321 & 15 & 8.752 & 0.002 \\
			2 & 20.4 & 0.28 & 114002 & 30 & 3775 & 11 & 8.236 & 0.003 \\
			3 & 30.5 & 0.35 & 130668 & 60 & 2153 & 6 & 7.675 & 0.003 \\
			4 & 40.55 & 0.4 & 115211 & 90 & 1255 & 4 & 7.135 & 0.003 \\
			5 & 50.6 & 0.45 & 106440 & 140 & 735 & 2 & 6.6 & 0.003 \\
			6 & 60.8 & 0.49 & 114715 & 240 & 453 & 1 & 6.116 & 0.003 \\
			\hline
		\end{tabular}
	\end{center}
	\label{table_2}
\end{table}

\begin{table}[h!]
	\caption{Ослабление \ga-лучей в \textbf{алюминии}}
	\begin{center}
		\begin{tabular}{|c|c|c|c|c|c|c|c|c|}
			\hline
			№ & $ l $, мм & $ \sigma_l $, мм & $ N' $ & $ t $, с & $ N, \; с^{-1}$  & $ \sigma_N, \; с^{-1}$ & $ \ln N $ & $ \sigma_{\ln N} $ \\
			\hline
			1 & 19.9 & 0.2 & 159422 & 20 & 7946 & 20 & 8.98 & 0.003 \\
			2 & 39.7 & 0.28 & 159510 & 30 & 5292 & 13 & 8.574 & 0.003 \\
			3 & 59.8 & 0.35 & 142914 & 40 & 3548 & 9 & 8.174 & 0.003 \\
			4 & 79.7 & 0.4 & 132888 & 60 & 2190 & 6 & 7.692 & 0.003 \\
			5 & 99.8 & 0.45 & 145262 & 90 & 1589 & 4 & 7.371 & 0.003 \\
			\hline
		\end{tabular}
	\end{center}
	\label{table_3}
\end{table}

\begin{table}[h]
	\caption{Ослабление \ga-лучей в \textbf{древесине}}
	\begin{center}
		\begin{tabular}{|c|c|c|c|c|c|c|c|c|}
			\hline
			№ & $ l $, мм & $ \sigma_l $, мм & $ N' $ & $ t $, с & $ N, \; с^{-1}$  & $ \sigma_N, \; с^{-1}$ & $ \ln N $ & $ \sigma_{\ln N} $ \\
			\hline
			1 & 19.5 & 0.2 & 237645 & 20 & 11857 & 24 & 9.381 & 0.002 \\
			2 & 39.5 & 0.28 & 231631 & 20 & 11557 & 24 & 9.355 & 0.002 \\
			3 & 59.5 & 0.35 & 227327 & 20 & 11341 & 24 & 9.336 & 0.002 \\
			4 & 79.5 & 0.4 & 222510 & 20 & 11101 & 24 & 9.315 & 0.002 \\
			5 & 98.5 & 0.45 & 219907 & 20 & 10970 & 23 & 9.303 & 0.002 \\
			6 & 118.5 & 0.49 & 215922 & 20 & 10771 & 23 & 9.285 & 0.002 \\
			\hline
		\end{tabular}
	\end{center}
	\label{table_4}
\end{table}


	
	
\end{document}