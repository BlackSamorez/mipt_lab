\section*{Ход работы, результаты}
\subsection{Измерение фона}
В ходе работы были проведены измерения для определения коэффициента поглащения трех металлов: алюминия (Al), свинца (Pb) и железа (Fe). Промежуток времени, на котором считалось число частиц в каждом наблюдении, сохранялся равным 10с. 

Первое измерение -- фон. Образец был закрыт специальным считом, при этом дачик показал приблизительно постоянные измерения. Результаты представлены в таблице \href{table:back}

\begin{table}[!ht]
    \centering
    \label{table:back}
    \begin{tabular}{|l|l|}
    \hline
        № измерения & $N_0$ \\ \hline
        1 & 336  \\ \hline
        2 & 309  \\ \hline
        3 & 317  \\ \hline
        4 & 329  \\ \hline
        5 & 320  \\ \hline
        6 & 311  \\ \hline
        7 & 318  \\ \hline
        8 & 334  \\ \hline
        9 & 308  \\ \hline
        10 & 331  \\ \hline
        среднее & 321 \\ \hline
        погрешность & $\pm$ 1 \\ \hline
    \end{tabular}
\end{table}

Полученное значение бедут вычитаться из всех последующих измерений, чтобы компенсировать фоновое излучение.

\subsection{Измерение Al}
Первая серия измерений проводилась с образцами из алюминия. Для регулирования толщины поглащающего/рассеивающего слоя использовались металлические бруски толщиной  $l_0 = 2.2\text{см}$. Результаты измерений представлены в таблице 1.

\begin{table}[!ht]
    \centering
    \label{table:Al}
    \caption{Измерения Al}
    \begin{tabular}{|l|l|l|l|l|}
    \hline
        $l_0$ Fe, мм & $\sum l_i$, мм & N & $\frac {l_0}{l}$ & $\ln( \frac {N} {N_0} )$  \\ \hline
        2,2 & 2,2 & 71526 & 1,000 & 1  \\ \hline
        2,2 & 4,4 & 48000 & 0,500 & 0,6696  \\ \hline
        2,2 & 6,6 & 31844 & 0,333 & 0,4427  \\ \hline
        2,2 & 8,8 & 20932 & 0,250 & 0,2895  \\ \hline
        2,2 & 11 & 14321 & 0,200 & 0,1966  \\ \hline
        2,2 & 13,2 & 9675 & 0,167 & 0,1314  \\ \hline
        3,2 & 16,4 & 6815 & 0,195 & 0,0912  \\ \hline
    \end{tabular}
\end{table}

По результатам измерений построим график в логарифмированных координатах. График представлен на рис. \ref{fig:Al_plot}. Из него получаем коэффициент $\mu_{Al} = 1.47 \pm 0.04$, что соответствует энергии $\gamma$-кванта, равной 0.6 МэВ.

\begin{figure}[h]
    \centering
    \includegraphics[width=1\textwidth]{Al_plot.png}
    \caption{График для Al}
    \label{fig:Al_plot}
\end{figure}

\subsection{Измерение Pb}
Вторая серия измерений проводилась аналогично с образцами из свинца. $l_0 = 0.47\text{см}$. Результаты измерений представлены в таблице 2.

\begin{table}[!ht]
    \centering
    \label{table:Pb}
    \caption{Измерения Pb}
    \begin{tabular}{|l|l|l|l|l|}
    \hline
        $l_0$ Fe, мм & $\sum l_i$, мм & N & $\frac {l_0}{l}$ & $\ln( \frac {N} {N_0} )$  \\ \hline
        0,47 & 0,47 & 58182 & 1,000 & 1  \\ \hline
        0,47 & 0,94 & 32415 & 0,500 & 0,5547  \\ \hline
        0,47 & 1,41 & 18851 & 0,333 & 0,3202  \\ \hline
        0,47 & 1,88 & 11189 & 0,250 & 0,1878  \\ \hline
        0,47 & 2,35 & 6628 & 0,200 & 0,1090  \\ \hline
        0,47 & 2,82 & 4064 & 0,167 & 0,0647  \\ \hline
    \end{tabular}
\end{table}

По результатам измерений построим график в логарифмированных координатах. График представлен на рис. \ref{fig:Pb_plot}. Из него получаем коэффициент $\mu_{Pb} = 0.249 \pm 0.01$, что соответствует энергии $\gamma$-кванта, равной 0.6 МэВ.

\begin{figure}[h]
    \centering
    \includegraphics[width=1\textwidth]{Pb_plot.png}
    \caption{График для Pb}
    \label{fig:Pb_plot}
\end{figure}

\subsection{Измерение Fe}
Третья серия измерений проводилась с образцами из железа. $l_0 = 1.01\text{см}$. Результаты измерений представлены в таблице 3.

\begin{table}[!ht]
    \centering
    \label{table:Fe}
    \caption{Измерения Fe}
    \begin{tabular}{|l|l|l|l|l|}
    \hline
        $l_0$ Fe, мм & $\sum l_i$, мм & N & $\frac {l_0}{l}$ & $\ln( \frac {N} {N_0} )$  \\ \hline
        1,01 & 1,01 & 60826 & 1,000 & 1  \\ \hline
        1,01 & 2,02 & 34441 & 0,500 & 0,5639  \\ \hline
        1,01 & 3,03 & 19734 & 0,333 & 0,3208  \\ \hline
        1,01 & 4,04 & 11367 & 0,250 & 0,1826  \\ \hline
        1,01 & 5,05 & 6751 & 0,200 & 0,1063  \\ \hline
        1,01 & 6,06 & 4072 & 0,167 & 0,0620  \\ \hline
    \end{tabular}
\end{table}

По результатам измерений построим график в логарифмированных координатах. График представлен на рис. \ref{fig:Fe_plot}. Из него получаем коэффициент $\mu_{Al} = 0.691 \pm 0.01$, что соответствует энергии $\gamma$-кванта, равной 0.5 МэВ.

\begin{figure}[h]
    \centering
    \includegraphics[width=1\textwidth]{Fe_plot.png}
    \caption{График для Fe}
    \label{fig:Fe_plot}
\end{figure}

\newpage
\subsection{Выводы}
В ходе выполнения лабораторной работы были проведены измерения коэффициента ослабления $\mu$ для гамма-квантов при прохождении через различные материалы: алюминий (Al), свинец (Pb) и железо (Fe).

Для алюминия был получен коэффициент ослабления $\mu_{\text{Al}} = 1.47 \pm 0.04$, что соответствует энергии $\gamma$-кванта, равной приблизительно 0.6 МэВ. Для свинца получен коэффициент $\mu_{\text{Pb}} = 0.249 \pm 0.01$, что также соответствует энергии $\gamma$-кванта около 0.6 МэВ. Наконец, для железа получен коэффициент $\mu_{\text{Fe}} = 0.691 \pm 0.01$, что соответствует энергии $\gamma$-кванта около 0.5 МэВ.

В действительности изучаемое излучение испускает $^{137}Cs$ по каналу $\beta^{-}$, что означает табличное значение энергии гамма-кванта в $0.6617МэВ$. Таким образом, отклонение составляет менее 10\%, что достаточно хорошо сходится с теорией.  

Результаты эксперимента согласуются с теорией ослабления гамма-квантов в веществе и позволяют оценить энергии используемых $\gamma$-квантов. Погрешности измерений в пределах указанных значений свидетельствуют о хорошей точности проведенных измерений.