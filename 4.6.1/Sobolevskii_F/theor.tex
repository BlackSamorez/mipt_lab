\section{Аннотация}
В данной работе были изучены электромагнитные волны миллиметрового диапазона, их проникающая способность и интерференция. Для исследования явления интерференции были применены два различных интерферометра. На основе полученных данных экспериментально определена длина волны излучения, а также коэффициент преломления диэлектрической пластины. Также экспериментально проверен закон Малюса.

\section{Теоретические сведения}
Когерентные электромагнитные волны миллиметрового диапазона, как и световые волны, интерферируют между собой. Если в некоторой точке пространства происходит суперпозиция двух когерентных одинаково поляризованных волн с интенсивностями $I_1$ и $I_2$ и с разностью фаз $\varphi$, то интенсивность $I$ результирующего колебания определяется соотношением 
\begin{equation}\label{interf}
    I = I_1 + I_2 + 2\sqrt{I_1 I_2}\cos \varphi
\end{equation}
Интенсивность максимальна при $\varphi = 2\pi m$, минимальна при $\varphi = (2m+1)\pi$ ($m = 0, 1, 2, ...$). 

Для получения когерентных электромагнитных волн чаще всего применяют установки, с помощью которых излучение от одного источника раскладывается на две составляющие с некоторой разностью хода между ними. Оптическая разность хода $\Delta$ и разность фаз связаны соотношением
\begin{equation}\label{pathDiff}
   \varphi = 2\pi\frac{\Delta}{\lambda}. 
\end{equation}
В настоящей работе оптическая разность хода определяется геометрией установок, а также оптическими плотностями сред, через которые проходит свет: оптический ход луча прямо пропорционален коэффициенту преломления среды.

В случае, когда линейно поляризованная электромагнитная волна падает на зеркало, интенсивность отражённой волны может уменьшаться в зависимости от угла падения. Эта зависимость определяется \textit{законом Малюса}: результирующая интенсивность $I$ зависит от интенсивности падающей волны $I_0$ и угла падения $\alpha$ как
\begin{equation}\label{Malus}
    I = I_0\cos^2 \alpha.
\end{equation}